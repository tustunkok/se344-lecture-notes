%%%%%%%%%%%%%%%%%%%%%%%%%%%%%%%%%%%%%%%%%
% kaobook
% LaTeX Template
% Version 1.3 (December 9, 2021)
%
% This template originates from:
% https://www.LaTeXTemplates.com
%
% For the latest template development version and to make contributions:
% https://github.com/fmarotta/kaobook
%
% Authors:
% Federico Marotta (federicomarotta@mail.com)
% Based on the doctoral thesis of Ken Arroyo Ohori (https://3d.bk.tudelft.nl/ken/en)
% and on the Tufte-LaTeX class.
% Modified for LaTeX Templates by Vel (vel@latextemplates.com)
%
% License:
% CC0 1.0 Universal (see included MANIFEST.md file)
%
%%%%%%%%%%%%%%%%%%%%%%%%%%%%%%%%%%%%%%%%%

%----------------------------------------------------------------------------------------
%	PACKAGES AND OTHER DOCUMENT CONFIGURATIONS
%----------------------------------------------------------------------------------------

\immediate\write18{makeindex -s nomencl.ist -o "\jobname.nls" "\jobname.nlo"}
\documentclass[
	a4paper, % Page size
	fontsize=10pt, % Base font size
	twoside=true, % Use different layouts for even and odd pages (in particular, if twoside=true, the margin column will be always on the outside)
	%open=any, % If twoside=true, uncomment this to force new chapters to start on any page, not only on right (odd) pages
	%chapterentrydots=true, % Uncomment to output dots from the chapter name to the page number in the table of contents
	numbers=noenddot, % Comment to output dots after chapter numbers; the most common values for this option are: enddot, noenddot and auto (see the KOMAScript documentation for an in-depth explanation)
]{kaobook}

% Choose the language
\ifxetexorluatex
	\usepackage{polyglossia}
	\setmainlanguage{english}
\else
	\usepackage[english]{babel} % Load characters and hyphenation
\fi
\usepackage[english=british]{csquotes}	% English quotes

\usepackage{kaobiblio}
% Load mathematical packages for theorems and related environments
\usepackage[framed=true]{kaotheorems}
% Load the package for hyperreferences
\usepackage{kaorefs}

\usepackage{float}
\usepackage{adjustbox}
\usepackage{pifont}
\usepackage{makecell}
\usepackage{dsfont}
\usepackage{threeparttable}
\usepackage{tkz-graph}
\usepackage{collcell}
\usepackage{tabularx}
\usepackage{menukeys}
\usepackage{amsmath}

\addbibresource{references.bib}

\renewcommand\theadfont{\bfseries}
\newcommand{\cmark}{\ding{51}}
\newcommand{\xmark}{\ding{55}}
\def\CC{{C\nolinebreak[4]\hspace{-.05em}\raisebox{.4ex}{\tiny\normalfont\bfseries ++}}}
\newcommand{\includepic}[1]{\includegraphics[width=3.5cm,keepaspectratio]{images/#1}}
\newcolumntype{i}{@{\hspace{1ex}}>{\collectcell\includepic}c<{\endcollectcell}}

\makeindex[intoc=true,columns=3,title=Index,options= -s index-style.ist]

\makeglossaries % Make LaTeX produce the files required to compile the glossary
\input{glossary.tex}

\makenomenclature % Make LaTeX produce the files required to compile the nomenclature

\counterwithin*{sidenote}{chapter}

\begin{document}
    \titlehead{Atılım University\\Department of Software Engineering}
    
    \title[Practical Software Testing]{Practical Software Testing}
    
    \author{Ali Yazıcı\thanks{E-mail: ali.yazici@atilim.edu.tr} \and Tolga Üstünkök \and Özge Tekin \and Zeynep Yaren Oğuz}
    
    \date{January 2023}
    
    \publishers{Department of Software Engineering\\Atılım University}
    
    \frontmatter
    
    %----------------------------------------------------------------------------------------
    %	COPYRIGHT PAGE
    %----------------------------------------------------------------------------------------
    
    \makeatletter
    \uppertitleback{\@titlehead} % Header
    
    \lowertitleback{
    	\textbf{Disclaimer}\\
    	You can edit this page to suit your needs. For instance, here we have a no copyright statement, a colophon and some other information. This page is based on the corresponding page of Ken Arroyo Ohori's thesis, with minimal changes.
    	
    	\medskip
    	
    	\textbf{No copyright}\\
    	\cczero\ This book is released into the public domain using the CC0 code. To the extent possible under law, I waive all copyright and related or neighbouring rights to this work.
    	
    	To view a copy of the CC0 code, visit: \\\url{http://creativecommons.org/publicdomain/zero/1.0/}
    	
    	\medskip
    	
    	\textbf{Colophon} \\
    	This document was typeset with the help of \href{https://sourceforge.net/projects/koma-script/}{\KOMAScript} and \href{https://www.latex-project.org/}{\LaTeX} using the \href{https://github.com/fmarotta/kaobook/}{kaobook} class.
    	
    	The source code of this book is available at:\\\url{https://github.com/fmarotta/kaobook}
    	
    	(You are welcome to contribute!)
    	
    	\medskip
    	
    	\textbf{Publisher} \\
    	First printed in May 2019 by \@publishers
    }
    \makeatother
    
    %----------------------------------------------------------------------------------------
    %	DEDICATION
    %----------------------------------------------------------------------------------------
    
    \dedication{
    	“In many ways, being a good tester is harder than being a good developer because testing requires not only a very good understanding of of the development process and its products, but it also demands an ability to anticipate likely faults and errors.”
    	\flushright --John D. McGregor
    	\flushright {\small Practical Guide to Testing Object-Oriented Software\\Addison Wesley, 2001}
    }
    
    %----------------------------------------------------------------------------------------
    %	OUTPUT TITLE PAGE AND PREVIOUS
    %----------------------------------------------------------------------------------------
    
    % Note that \maketitle outputs the pages before here
    
    \maketitle
    
    %----------------------------------------------------------------------------------------
    %	PREFACE
    %----------------------------------------------------------------------------------------
    
    \input{chapters/preface.tex}
    \index{preface}
    
    %----------------------------------------------------------------------------------------
    %	TABLE OF CONTENTS & LIST OF FIGURES/TABLES
    %----------------------------------------------------------------------------------------
    
    \begingroup % Local scope for the following commands
    
    % Define the style for the TOC, LOF, and LOT
    %\setstretch{1} % Uncomment to modify line spacing in the ToC
    %\hypersetup{linkcolor=blue} % Uncomment to set the colour of links in the ToC
    \setlength{\textheight}{230\hscale} % Manually adjust the height of the ToC pages
    
    % Turn on compatibility mode for the etoc package
    \etocstandarddisplaystyle % "toc display" as if etoc was not loaded
    \etocstandardlines % "toc lines" as if etoc was not loaded
    
    \tableofcontents % Output the table of contents
    
    \listoffigures % Output the list of figures
    
    % Comment both of the following lines to have the LOF and the LOT on different pages
    \let\cleardoublepage\bigskip
    \let\clearpage\bigskip
    
    \listoftables % Output the list of tables
    
    \listoflistings
    
    \endgroup
    
    %----------------------------------------------------------------------------------------
    %	MAIN BODY
    %----------------------------------------------------------------------------------------

    \mainmatter
    \setchapterstyle{kao} % Choose the default chapter heading style
    
    \pagelayout{wide} % No margins
    \addpart{Software Testing}
    \pagelayout{margin} % Restore margins
    
    \setchapterimage[7.5cm]{images/matt-hardy-6ArTTluciuA-unsplash}
\setchapterpreamble[u]{\margintoc}
\chapter{Basic Concepts of Testing}
\section{Software Quality and Testing}
Software testing is the process of evaluating and verifying that the complete software product or parts of it work and perform as specified by the customer's requirements (SRSD). This process helps in detecting the bugs during the development stage, hence reducing the cost of development and improving performance. In addition, the testing process plays a vital role in achieving and assessing the quality of a software
product \autocite{friedman1995software}. 

Quality assessment activities are discussed in two categories, namely, static analysis and dynamic analysis. Static analysis is based on SRSDs, Software Design Documents (SDD), models, and typically source code. Code review, walkthrough, and inspection are the most common static analysis techniques. The actual execution of the code is not involved in static analysis but rather the source code, byte code, or application binaries are included in the analysis. Static analysis is the more thorough approach and may also prove to be more cost-efficient with the ability to detect bugs at an early phase of the SDLC. Dynamic analysis of a software product involves the actual execution of the code in an attempt to detect bugs. During the dynamic analysis functionality and performance of a program are also observed. Static analysis\index{static analysis} and dynamic analysis\index{dynamic analysis} approaches are complementary to improve the quality of the software product and can be performed in a synchronized and/or planned manner.

In software testing, verification\index{verification} and validation\index{validation} (V\&V) are two similar abstract concepts. Software Engineering standards known as IEEE/ISO/IEC 24765:2017\sidenote{The standard can be found in \url{https://www.iso.org/standard/71952.html}.} (ISO/IEC/IEEE International Standard - Systems and Software Engineering - Vocabulary) defines verification and validation as the process of determining whether the requirements for a system or component are complete and correct, the products of each development phase fulfills the requirements or conditions imposed by the previous phase, and the final system or component complies with specified requirements. In the software world, it is common to define verification process as an answer to the questions, "Are we building the product right?", and validation process as an answer to "Am I building the right product?".

Code reviews\index{code reviews}, inspections\index{inspections}, walkthroughs\index{walkthrough}, and low level tests such as unit testing\index{unit testing}, integration testing\index{integration testing}, static checking\index{static checking} of SRSD and SDD and files are some of the activities for verification, and applying various testing techniques such as acceptance testing, usability testing and other non-functional testing are some of the validation activities. Some of the testing methods such as beta testing, regression testing can be listed as activities at the intersection of V\&V. In this book, the testing activities mentioned here will be explained in detail.

\section{Error, Fault, Defect and Failure}
Error, fault (bug), defect, and failure are very common terms used in software testing. And these terms are sometimes used interchangeably, albeit incorrectly. To avoid confusion and establish their consistent use in software testing area, ISO/IEC/IEEE 24765:2017 “Systems and software engineering — Vocabulary” standard will be used. The definitions taken from this standard and used in this book are given below.

\paragraph{Error}
\index{error}(1) human action that produces an incorrect result, (2) difference between a computed, observed, or measured value or condition and the true, specified, or theoretically correct value or condition, (3) erroneous state of the system.

\paragraph{Fault (bug)}
\index{fault (bug)}(1) manifestation of an error in software, (2) incorrect step, process, or data definition in a computer program, (3) situation that can cause errors to occur in an object, (4) defect in a hardware device or component, (5) defect in a system or a representation of a system that if executed/activated could potentially result in an error.

\paragraph{Defect}
\index{defect}(1) imperfection or deficiency in a work product where that work product does not meet its requirements or specifications and needs to be either repaired or replaced, (2) an imperfection or deficiency in a project component where that component does not meet its requirements or specifications and needs to be either repaired or replaced, (3) generic term that can refer to either a fault (cause) or a failure (effect). 

\paragraph{Failure}
\index{failure}(1) termination of the ability of a system to perform a required function or its inability to perform within previously specified limits; an externally visible deviation from the system's specification.

\section{Objectives of Testing}
The foremost objective of Software Testing is to improve the quality of the product by detecting errors and removing faults made during the development phase of SDLC. 

Providing quality products is the ultimate goal of testing. Customer satisfaction is ensured and as a result, the competitive power of the software company is increased.

In order to comply with these goals, software test teams are established and the product is tested using different methods at each stage (level) of development. These methods of software testing will be elaborated in the chapters that follow.

\section{Test Levels in the V-Model}
Software testing is applied at different levels of development. Considering the so-called V-model\index{V-model}, the levels of testing can be identified as unit (component), integration, system, and acceptance testing. 

While all software development methodologies have different approaches, product development tasks will be performed in the order of (1) requirements gathering and analysis, (2) high-level design, (3) comprehensive design, and (4) coding. 
These tasks match the software testing levels mentioned above as shown in \reffig{fig:v-model}. Unit, Integration and system tests are usually performed by the developers, and acceptance tests by the customer in collaboration with the software developers.

\begin{figure}[!ht]
    \includegraphics{images/v-model.png}
    \caption{Validation and Verification with the V-model}
    \labfig{fig:v-model}
\end{figure}

In unit testing, individual units/components such as functions, procedures (in developments  with procedural languages) and  methods and classes (in developments with object-oriented languages) are tested. Following the successful completion of the unit tests, related units and components are systematically brought together and subjected to integration tests. The objective of integration testing\index{integration testing} is to build a stable product configuration ready for system tests. 

System level testing\index{system level testing} includes a wide spectrum of testing, such as functionality testing, security testing, robustness testing, load testing, stability testing, stress testing, performance testing, and reliability testing \autocite{naik2011software}. After the successful completion of system level testing, the product is delivered to the customer. The customer performs a series of tests, known as acceptance testing. The main objective of acceptance testing\index{acceptance testing} is to confirm that the system meets the agreed-upon criteria (functional and non-functional requirements), identify and resolve discrepancies, if there are any and determine the readiness of the system for live operations. The final acceptance of a system for deployment is conditioned upon the outcome of the acceptance testing. The acceptance test team produces an acceptance test report which outlines the acceptance conditions. 

Another test level applied at the first three levels is the regression test\index{regression testing}. This test is performed whenever a part of the system (unit, component, or module) is replaced or modified. The main idea of regression testing is to determine if the change creates a side effect (new bugs) on other parts of the software that have already been completed and passed the tests. 

In regression testing, new tests are not designed. Instead, tests are selected, prioritized, and executed from the existing pool of test cases to ensure that nothing is broken in the new version of the software \autocite{naik2011software}. 

\section{White-box and Black-box Testing}
White-box (glass box, clear box, transparent box) testing is a low-level testing technique which deals with the internal working of the software system. During white-box testing, assignment and predicate statements, and the branches and execution paths are considered and checked for possible faults. On the contrary to white-box testing\index{white-box testing}, black-box (data-driven, functional) testing\index{black-box testing} does not care about the internal workings of the code but deals with the overall behaviour and functionality of the parts of a code and as such can be applied at all levels of testing.

White-box testing is suitable at unit and integration levels, whereas black-box testing is ideal at system and acceptance levels. Programming knowledge is necessary for white-box testing. On the contrary, for black-box testing, this is not essential.

\section{Testing Activities}
Main testing activities are planning, analysis, design, implementation, and  execution.  Testing team (or developer) comes up with some requirements to be tested. Test planning considers the team management, cost and the duration of the testing process, and some other related quality assurance metrics. Analysis activity involves understanding the nature of tests to be conducted, complexity of the test, and the risks involved in the execution of the tests. Depending on the tasks to be completed, and the input data required, tests are designed. At the test implementation step tests are chosen and prioritized. Software test execution is the last activity to perform. Test results are systematically collected and reported along with the required testing metrics.

A test plan can have more than one test suite. Each test suite has one or more related test cases. A test case (TC) is usually\index{test suite} defined as a pair of \textbf{<input, expected outcome>}. If a code\index{test case} under test is to compute the absolute value of a real number, say, x, then one can list three test cases as follows:
\begin{itemize}[nosep]
    \item TC1: <0.0, 0.0>
    \item TC2: <-26.4, +26.4>
    \item TC3: <+14.0, +14.0>
\end{itemize}
A test case is prepared using the requirements and functional specifications, source code, and input/output domains. Expected outcome can be as simple as a single numerical value, a more complex data type like audio, photo or video, a state change or a set of values which needs to be interpreted together for the outcome to be valid \autocite{naik2011software}.

Another important concept in testing is test oracle. A test oracle\index{test oracle} (or simply oracle) is a mechanism for determining whether a test has passed or failed. An oracle will respond with a pass or a fail verdict on the acceptability of any test sequence for which it is defined. When executing a test case, the correctness of the implementation is confirmed by an oracle.

In most testing environments, test oracle is the human who designs the test \autocite{bertolino1996use}. A simple example of a test oracle is given below\sidenote{Detailed information about test oracle can be found in the link: \url{https://bit.ly/3K9iqZj}}. 

Given a list of 3 integers, 11, 23, and 19, what is the expected value from a function, say, computeMax(), which returns the maximum? In this case, the test oracle is easy to identify.

\begin{itemize}
    \item \lstinline!int expected = 23!; 
    \item \lstinline!int actual = computeMax(11, 23, 19);! \lstinline!computeMax! is a method which returns the maximum of the three numbers.
    \item \lstinline!assertEquals(expected, actual);!
\end{itemize}

Note that \lstinline!assertEquals()!\index{assertEquals()} is a method in the JUnit testing library to compare two objects for equality. This method will be introduced during laboratory sessions.

\section{Problems}
\begin{enumerate}
    \item Create test cases for a computer program to find the positive square root of a real number.
    \item Discuss the difference between white-box and black-box testing strategies.
    \item What is a stateless software system? Give an example.
    \item ATM is an example of a state-oriented system. Create a test case for withdrawing money from a bank account. First define the input, and write down the expected output(s) during the transaction.\\
    \emph{Hint. In this case, a test case will contain more than one <input, expected> pairs and input may be choosing an item from a menu.}
    \item What is agile software testing? 
    \item What are the main differences between classical testing and agile testing approaches?
    \item What is Test Driven Development (TDD)?
    \item What is a test oracle? Explain by using a simple testing activity.
    \item Differentiate between verification and validation. Describe various verification and validation methods.
    \item What is the main difference between inspections and walkthroughs?
\end{enumerate}
    \setchapterimage[7.5cm]{images/daniel-sinoca-AANCLsb0sU0-unsplash}
\setchapterpreamble[u]{\margintoc}
\chapter{Unit (Component) Testing}
\section{Unit Testing Basics}
First level of testing is unit testing. Unit (component) is a manageable piece of code which may produce a result, perform a computation and return its result(s) to its caller, or may call some other function or method. In procedural programming, a unit could be an individual function or procedure (subroutine). In an object-oriented setting, a unit could be an individual method or even a class. 

Unit testing is a low-level testing technique by means of which individual units are tested to determine if they produce expected outcomes or not. All assignment statements and predicates (logical expressions) in the code must be run, as well as all paths, while achieving the expected output with the unit test. In general, unit testing activity can be described as a type of testing process used to verify and validate a specific unit of the code for its correctness to cover the coding standards, functionality, integration, code coverage, security features, compatibility, performance, and so on. Unit testing is done by the person who develops the unit. The following unit examples are written in C and Java.

\begin{example}
Write a C function \lstinline!solveQuadratic! to solve the quadratic $ax^2+bx+c=0$ for the two real roots x1, and x2. Coefficients of the quadratic polynomial is to be provided by the calling program.

\begin{lstlisting}[language=C, caption={A C function that solves the quadratic equation for the two real roots.}]
double solveQuadratic(double a, double b, double c) {
	double disc, x1, x2;
	disc = b * b - 4.0 * a * c;
	if (disc < 0) {
		printf("Discriminant = %f7.4 :no real roots!", disc);
	} else {
		x1 = (-b + sqrt(disc) / (2.0 * a);
		x2 = (-b - sqrt(disc)) / (2.0 * a);
		printf("x1 = %f7.4, x2 = %f7.4\n",x1, x2);
	}
	return (0);
}
\end{lstlisting}
\end{example}


\begin{example}
\labexample{ex22}
Binary search is a very effective technique to search an element in a set of ordered elements stored in an array. The following C program is an implementation borrowed from \url{https://www.tutorialspoint.com/compile_c_online.php}. Some bugs are  intentionally inserted into the code. The reader is encouraged to review this example, identify and correct errors, and run it again. 

\begin{lstlisting}[language=C, caption={Binary search a C implementation.}]
// Iterative C Program for Binary Search (GNU GCC v7.1.1)
// Adapted from https://www.tutorialspoint.com/codingground.htm
// 3 severe bugs are inserted! 
#include <stdio.h>
int BinarySearch(int array[], int left, int right, int element){
   while (left <= right){
      int middle = left + (right + left)/2;
      if (array[middle] == element)
         return middle;
      if (array[middle] < element)
         left = middle - 1;
      else
         right = middle + 1;
   }
   return -1;
}
int main(void){
   int array[] = {11, 14, 17, 21, 28, 43, 41, 52, 56, 70, 76, 82, 99};
   int n = 13;
   int element = 99;
   int index = BinarySearch(array, 0, n-1, element);
   if(index == -1 ) {
      printf("Element not found!");
   }
   else {
      printf("Element found at position : %d",index);
   }
   return 0;
}
\end{lstlisting}
\end{example}
Unit testing is conducted in two different but complementary phases, namely, static unit testing and dynamic unit testing. In static unit testing, the unit under the test is examined, without actually executing it, to infer conclusions about its functionality, and causes of some possible faults. In dynamic unit testing, a program unit is actually executed and its outcomes are recorded and analysed.

\section{Static Unit Testing – Code Reviews}
Static unit testing\index{static unit testing} is accomplished by using several code review techniques. Code review is not restricted to units. In practice, during this process, completed parts of the code (unit, module, or component) are reviewed using inspection and walkthrough techniques. Inspection\index{inspection} is a step by step peer group review of a work product (including a document), with each step checked against predetermined criteria \autocite{fagan1999design}. Walkthrough\index{walkthrough} is a similar review technique in which the developer leads the team through a manual or simulated execution of the product using predefined scenarios \autocite{yourdon1989structured}.

The main goal of the code review is to review the code in an attempt to find the errors in the code before it is passed on to another activity (e.g., integration testing). The code review should be performed in a systematic manner by executing planned activities. To facilitate the process a review team is established. Review team is usually made up of members who will assume the roles of facilitator, code developer (author), presenter, record keeper, reviewer, and observer.
The general guidelines for performing code review consists of six steps \autocite{naik2011software}.

\begin{description}
    \item[Readiness] Unit developer assures that the unit is ready for inspection. 
    \item[Preparation] Each reviewer goes over the code and the related documents. A list of potential CR is prepared by the reviewers.
    \item[Examination and Producing a Change Request (CR) Document] In this step, the author, presenter, the record keeper, and the moderator are involved and, if needed, a CR document is prepared and passed on to the next step.
    \item[Rework] The CR list that needs to be resolved is forwarded to the developer.
    \item[Validation] The rework done by the developer is validated by one of the software team members. The validation process involves checking the improvements in the code as suggested by the other group members. 
    \item[Reporting] If all CRs are fulfilled, a detailed report is prepared and shared with the team members.
\end{description}

To facilitate the code review process, software teams use predefined checklists. One such checklist is given below\sidenote[][-1.5cm]{This document\\\url{https://github.com/mgreiler/code-review-checklist} is protected under the MIT license}. 

\paragraph{Implementation}
\begin{enumerate}
    \item Does this code change do what it is supposed to do?
    \item Can this solution be simplified?
    \item Does this change add unwanted compile-time or run-time dependencies?
    \item Was a framework, API, library, service used that should not be used?
    \item Was a framework, API, library, service not used that could improve the solution?
    \item Is the code at the right abstraction level?
    \item Is the code modular enough?
    \item Would you have solved the problem in a different way that is substantially better in terms of the code’s maintainability, readability, performance, security?
    \item Does similar functionality already exist in the codebase? If so, why isn't this functionality reused?
    \item Are there any best practices, design patterns, or language-specific patterns that could substantially improve this code?
    \item Does this code follow Object-Oriented Analysis and Design Principles, like the Single Responsibility Principle, Open-close principle, Liskov Substitution Principle, Interface Segregation, Dependency Injection?
\end{enumerate}

\paragraph{Logic Errors and Bugs}
\begin{enumerate}
  \setcounter{enumi}{11}
  \item Can you think of any use case in which the code does not behave as intended?
  \item Can you think of any inputs or external events that could break the code?
\end{enumerate}
 
\paragraph{Error Handling and Logging} 
\begin{enumerate}
  \setcounter{enumi}{13}
  \item Is error handling done the correct way?
  \item Should any logging or debugging information be added or removed?
  \item Are error messages user-friendly?
  \item Are there enough log events and are they written in a way that allows for easy debugging?
\end{enumerate}

\paragraph{Usability and Accessibility}
\begin{enumerate}
 \setcounter{enumi}{17}
  \item Is the proposed solution well designed from a usability perspective?
  \item Is the API well documented?
  \item Is the proposed solution (UI) accessible?
  \item Is the API/UI intuitive to use?
\end{enumerate}
 
\paragraph{Ethics and Morality}
\begin{enumerate}
 \setcounter{enumi}{21}
 \item Does this change make use of user data in a way that might raise privacy concerns?
 \item Does the change exploit behavioral patterns or human weaknesses?
 \item Might the code, or what it enables, lead to mental and physical harm for (some) users?
 \item If the code adds or alters ways in which people interact with each other, are appropriate measures in place to prevent/limit/report harassment or abuse?
 \item Does this change lead to an exclusion of a certain group of people or users?
 \item Does this code change introduce any algorithm, AI or machine learning bias?
 \item Does this code change introduce any gender/racial/political/religious/ableist bias?
\end{enumerate}
 
\paragraph{Testing and Testability}
\begin{enumerate}
 \setcounter{enumi}{28}
 \item Is the code testable?
 \item Does it have enough automated tests (unit/integration/system tests)?
 \item Do the existing tests reasonably cover the code change?
 \item Are there some test cases, input, or edge cases that should be tested in addition?
\end{enumerate}
 
\paragraph{Dependencies} 
\begin{enumerate}
 \setcounter{enumi}{32}
  \item If this change requires updates outside of the code, like updating the documentation, configuration, readme files, was this done?
  \item Might this change have any ramifications for other parts of the system, backward compatibility?
\end{enumerate}

\paragraph{Security and Data Privacy}
\begin{enumerate}
 \setcounter{enumi}{34}
  \item Does this code open the software for security vulnerabilities?
  \item Are authorization and authentication handled in the right way?
  \item Is sensitive data like user data, credit card information securely handled and stored? Is the right encryption used?
  \item Does this code change reveal some secret information (like keys, usernames, etc.)?
  \item If code deals with user input, does it address security vulnerabilities such as cross-site scripting, SQL injection, does it do input sanitization and validation?
  \item Is data retrieved from external APIs or libraries checked accordingly?
\end{enumerate}

\paragraph{Performance}
\begin{enumerate}
 \setcounter{enumi}{40}
  \item Do you think this code change will impact system performance in a negative way?
  \item Do you see any potential to improve the performance of the code?
  \end{enumerate}
	
\paragraph{Readability}
\begin{enumerate}
 \setcounter{enumi}{42}
   \item Was the code easy to understand?
   \item Which parts were confusing to you and why?
   \item Can the readability of the code be improved by smaller methods?
   \item Can the readability of the code be improved by different function/method or variable names?
   \item Is the code located in the right file/folder/package?
   \item Do you think certain methods should be restructured to have a more intuitive control flow?
   \item Is the data flow understandable?
   \item Are there redundant comments?
   \item Could some comments convey the message better?
   \item Would more comments make the code more understandable?
   \item Could some comments be removed by making the code itself more readable?
    \item Is there any commented out code?
\end{enumerate}

\paragraph{Experts Opinion}
\begin{enumerate}
 \setcounter{enumi}{54}
   \item Do you think a specific expert, like a security expert or a usability expert, should look over the code before it can be committed?
   \item Will this code change impact different teams? Should they have a say on the change as well? 
\end{enumerate}

The code review checklist list above is comprehensive enough to be used at different levels of testing. However, for unit testing, it would be more appropriate to use a modified and shortened version of the list.

\section{Dynamic Unit Testing}
In dynamic unit testing\index{dynamic unit testing}, the unit under investigation is executed in order to detect the errors. An execution environment is created by writing a test driver\index{test driver} and compiling it together with the actual unit. If the unit under the test invokes (calls) some other units, dummy ones called stubs\index{stubs} are created to emulate the called units.

\marginnote[-5cm]{
	\begin{kaobox}[frametitle=Test Driver]
    A test driver is simply a program (main) that calls the unit under test. Test input data is provided by the test driver, unit is called, and the results are returned from the unit and assessed (test fails/passes) by the test driver.
    \end{kaobox}
}

It is important to note that, the test driver and the stubs are retained and reused in the future in regression testings if required. In \reffig{fig:unit-env}, a dynamic unit testing environment is illustrated.

\marginnote{
	\begin{kaobox}[frametitle=Stub]
    A stub as mentioned before, is a dummy function which replaces the actual unit called by the unit under test. A stub returns a preassigned or decided value (e.g., null, 0, or another expected result from the actual unit) so that the unit under test can resume its execution in a meaningful way.
    \end{kaobox}
}

\begin{figure}[!ht]
    \includegraphics{images/unit-env.png}
    \caption{Validation and verification with the V-model.}
    \labfig{fig:unit-env}
\end{figure}

Test case generation (manual or automatic) is one of the main issues of software testing research. In unit testing, the main concern is to detect the faults (bugs) in the assignment and control statements, predicates (logical expressions in the control statements), and the flow of data from one statement to another. Test input selection is mainly based on Control Flow Testing, Data Flow Testing, and Functional Program Testing techniques. These techniques will be discussed in the chapters to follow. 

\section{Mutation Testing}
Mutation testing is a white-box testing technique to measure the adequacy of test cases \autocite{demillo1978hints}. The idea in mutation testing\index{mutation testing} is to test the code by modifying/altering the program intentionally by applying one of the there mutation types, and observe how the code behaves when tested with the same test data.  Each modified code is known as a mutant.

There are three types of mutation:
\begin{description}
    \item[Value mutation] Change one constant to a larger value or to a smaller value.
    \item[Statement mutation] Change the statements by removing/deleting or modifying the line.
    \item[Decision mutation] Change the decisions/conditions to check for the design errors. Typically, one changes the arithmetic operators (e.g., -, +, /, *) and mutating all relational operators (==, !=, <, <=, >, >=) and logical operators (AND, OR , NOT)).
\end{description}

A mutant is said to be killed if the execution of a test case causes it to fail. Some mutants are equivalent to the given program producing the same output as the original one. The mutants still alive are survived mutants and need to be killed by adding additional tests to the test suite. A mutation score, $MS$, for a set of test cases is defined as the percentage of nonequivalent mutants killed by the test suite. If $K$ is the number of killed mutants, $N$ is the total number of mutants, and $E$ is the number of equivalent mutants, then $MS = \frac{k}{N - E} * 100\%$. The test suite is mutation adequate if the mutation score is 100\%.

\begin{example}
The following recursive C program prints the factorial of integers from 0 to 10. Create two mutants by applying value, and decision mutations and compute the mutation score.

\begin{lstlisting}[language=C, caption={A recursive C program to compute and print  0!, 1!,\ldots,10!}]
#include <stdio.h>
int fact (int);

main(){
	int k;
	for(k = 0; k <= 10; k++)
		printf("%2d! = %4d\n", k, fact(k));
}
int fact(int num){
	if (num <= 1)
		return (1);
	else
		return(num * fact(num - 1));
}
\end{lstlisting}

Output is a list of factorial values of numbers from 0 to 10.

\begin{lstlisting}[language=C, caption={Mutant 1: (Decision mutation - Change num <= 1 to num < 1)}]
#include <stdio.h>
int fact (int);

main(){
	int k;
	for(k = 0; k <= 10; k++)
		printf("%2d! = %4d\n", k, fact(k));
}
int fact(int num){
	if (num >= 1)
		return (1);
	else
		return(num * fact(num - 1));
}
\end{lstlisting}

\begin{lstlisting}[language=C, caption={Mutant 2: (Value mutation - Change fact(num –  1) to fact(num  - 100))}]
#include <stdio.h>
int fact (int);

main(){
	int k;
	for(k = 0; k <= 10; k++)
	printf("%2d! = %4d\n", k, fact(k));
}
int fact(int num){
	if (num <= 1)
		return (1);
	else
		return(num * fact(num - 100));
}
\end{lstlisting}

\begin{lstlisting}[language=C, caption={Mutant 3: (Decision mutant – Change  if (num <= 1) to if (num < 1))}]
#include <stdio.h>
int fact (int);

main(){
	int k;
	for(k = 0; k <= 10; k++)
	printf("%2d! = %4d\n", k, fact(k));
}
int fact(int num){
	if (num < 1)
		return (1);
	else
		return(num * fact(num - 1));
}
\end{lstlisting}

For the example above the test suite consists of 3 test cases:

\begin{itemize}
    \item Mutation1 - Test case 1: <no input, error message >
    \item Mutation2 - Test case 2: <no input, <1, 2, 3, 4, 5, 6, 7, 8, 9, 10> > (result is incorrect!)
    \item Mutation3 - Test case 3: <no input, correct result>
\end{itemize}

Test case 1, and Test case 2 fail and therefore said to be killed by these test cases ($K = 2$). However, Test case 3 produces correct result (factorial of numbers from 0 to 10 are printed correctly!) and is considered as an equivalent mutant ($E = 1$). The mutation score in this case is $\frac{2}{3 - 1} * 100 = 100\%$.
\end{example}

\begin{example}
The following C program prints the prime numbers from 2 to $n$, where the value of $n$ is the input. Create three mutants by applying value, and decision mutations and compute the mutation score.

\begin{lstlisting}[language=C, caption={A C program to compute and print primes from 2 to $n$.}]
//Prime numbers from 2 to n
#include<stdio.h>
main()
	printf("Enter the value of n>=2:\n");
	scanf("%d",&n);
	printf("Prime numbers are\n"); 
	for(i=2;i<=n;i++){
		int k=0;
		for(j=1;j<=i;j++)
			if(i%j==0)k++;		
		if(k==2)
			printf("%d ",i);
	}	
}
\end{lstlisting}

\begin{lstlisting}[language=C, caption={Mutant 1: (Decision mutant – Change $i \leq n$ to $i \geq n$)}]
//Prime numbers from 2 to n
#include<stdio.h>
main(){
	int i,j,n;
	printf("Enter the value of n>=2:\n");
	scanf("%d",&n);
	printf("Prime numbers are\n"); 
	for(i=2;i>=n;i++){
		int k=0;
		for(j=1;j<=i;j++)
			if(i%j==0)k++;		
		if(k==2)
		printf("%d ",i);
	}	
}
\end{lstlisting}

\begin{lstlisting}[language=C, caption={Mutant 2: (Decision mutant – Change $i \leq n$ to $i < n$)}]
//Prime numbers from 2 to n
#include<stdio.h>
main(){
	int i,j,n;
	printf("Enter the value of n>=2:\n");
	scanf("%d",&n);
	printf("Prime numbers are\n"); 
	for(i=2;i<n;i++){
		int k=0;
		for(j=1;j<=i;j++)
			if(i%j==0)k++;		
		if(k==2)
			printf("%d ",i);
	}	
}
\end{lstlisting}

\begin{lstlisting}[language=C, caption={Mutant 3: (Value mutant – Change $k == 2$ to $k == 3$)}]
//Prime numbers from 2 to n
#include<stdio.h>
main(){
	int i,j,n;
	printf("Enter the value of n>=2:\n");
	scanf("%d",&n);
	printf("Prime numbers are\n"); 
	for(i=2;i<=n;i++){
		int k=0;
		for(j=1;j<=i;j++)
			if(i%j!=0)k++;		
		if(k==3)
			printf("%d ",i);
	}	
}
\end{lstlisting}

For an input value of 19 for n, the original program produces the primes correctly from 2 to 19.

Output: 2, 3, 5, 7, 11, 13, 17, 19

Three test cases are as follows:
\begin{itemize}
    \item Mutation1 - Test case 1: <19, null > Test fails! No primes are generated. 
    \item Mutation2 - Test case 2: <19, <2, 3, 5, 7, 11, 17> > Test fails! 19 is missing from the list of primes.
    \item Mutation3 - Test case 3: <19, 5> Test fails! Only one prime number, 5, is generated.
\end{itemize}

All tests fail, and there are no equivalent mutants. The mutation score in this case is  $\frac{3}{3} * 100 = 100\%$.
\end{example}

\section{Debugging}
Debugging\index{debugging} is a systematic process of finding and fixing the number of faults (bugs), or defects, in a piece of code so that the software is behaving as expected. For small to medium size software, developers can use brute force approach to find faults by inserting print statements in the neighborhood of the possible fault(s). For large size projects this approach is so tedious and therefore the use dynamic debuggers provided in the Integrated Development Environments (IDEs) should be utilized. A screenshot from Dev-\CC IDE debugger of a program is shown in \reffig{fig:devcpp-debugger}. Breakpoints are placed at certain lines (red color), and the change in the values of a variable is observed by adding a watch. Next line tab takes the execution to the next statement and the changes are observed on the Debug tab of the IDE.

\begin{figure}[!ht]
    \includegraphics{images/dev-c-debugger.jpg}
    \caption{A screenshot of Dev-\CC Debugger.}
    \labfig{fig:devcpp-debugger}
\end{figure}
 
The reader is encouraged to learn the use of the debugger on DEV-\CC IDE by running and debugging the prime number generation program given above. Eclipse Java IDE used in the laboratory sessions, also offers similar debugging functionality. A screenshot from the debugging option of the Eclipse can be found in \reffig{fig:eclipse-debugger}.

\begin{figure}[!ht]
    \includegraphics{images/eclipse-debugger-screenshot.jpg}
    \caption{A screenshot of Eclipse Debugger.}
    \labfig{fig:eclipse-debugger}
\end{figure}

\section{Unit Test Automation}
Unit tests can be performed manually or with the help of automation tools. JUnit (Java), TestNG (Java), NUnit (.net languages), Jasmine (Javascript), Html Unit (HTML), and Simple Test (PHP) are among the most popular open source unit testing tools and frameworks. In the second part of this book, JUnit tool will be introduced within the Eclipse Java IDE.  

\section{Problems}
\begin{enumerate}
    \item Create a new code review checklist, by modifying the one given in this chapter to facilitate its use for units (functions, procedures, methods, or modules). 
    \item Write a recursive C program to sort n numbers by the Quicksort algorithm. Create value, decision and statement mutants, and test cases. Compute the mutation score based on your test cases.  
    \item Explain the difference between static and dynamic unit testing.
    \item What is regression testing? At what levels of testing it is utilized?
    \item In debugging, what is a breakpoint (toggle point) in a program segment?
    \item Use Dev-C++ IDE to debug the prime number generation program given in this chapter to learn the debugging process. Insert breakpoints (toggle points), and use add watch facility to trace the execution.
    \item Repeat the same using an Eclipse IDE for C programming language.
    \item Explain the difference between debugging and testing.
    \item What is static code analysis? Name a few popular open-source static code analyzer.
    \item Explain how mutation testing improves the quality of testing.
    
\end{enumerate}
    \setchapterimage[7.5cm]{images/jong-marshes-79mNMAvSORg-unsplash}
\setchapterpreamble[u]{\margintoc}
\chapter{Control Flow Testing}
\section{Introduction}
Control flow testing is a software testing  strategy that depicts the execution order of the assignment and control statements in a program unit such as a function. This strategy is implemented by developing test cases of a unit and executing and tracing the execution flow. Flow of execution of the assignment and I/O statements in a program unit is sequential and change whenever a control statement (if-then-else, for, switch/case and while) is encountered and executed. A program unit has an entry and an exit point. Commands in a program unit are executed from the command at the entry point to the command at the exit point, depending on the program flow. The sequence of these flow steps is defined as the program execution path. Depending on the number and complexity of control commands, multiple paths can occur in a unit. The paths in the program are shaped depending on the input values in the program unit.

A program unit can have multiple paths. Testing all paths with the input value that determines the path is not a very efficient approach and is costly. There are many approaches to increasing testing efficiency and minimizing its cost. The McCabe cyclomatic complexity analysis is one of them and will be discussed in the following sections.

The control flow testing steps are summarized below:
\begin{enumerate}
    \item Creation of control flow graph (CFG).
    \item Determination of the path to be tested according to the path selection criteria.
    \item Creation of necessary inputs and relevant test case for the determined path.
\end{enumerate}
Path selection criteria will be discussed in detail in the next subsections.

\section{Control Flow Graph (CFG)}
A control flow graph\index{control flow graph (CFG)} is a directed graph with an entry and an exit point. It is similar to a flowchart and is used to represent the overall flow in a unit. In a CFG, a rectangular node represents a sequential computation encompassing a set of statements in order, a decision node is used for branching (if-then-else), and a circle represents a merge point. A directed edge is used to connect nodes. In order to identify a path uniquely, each node is labeled with a unique integer. A decision node provides branching to the sides of the box by a True (T) or a Yes (Y) label. A complete execution path in a CFG is defined with a set of ordered nodes from 1 to N, where 1 and N are the labels of the entry point, and the exit point respectively. The basic nodes of a CFG is shown in \reffig{fig:cfg-31}.

\begin{figure}[!ht]
    \includegraphics{images/cfg-figure-3-1.png}
    \caption{Basic nodes (computation, decision and merge) in a CFG.}
    \labfig{fig:cfg-31}
\end{figure}

Using these nodes one can construct CFG elements for the basic programming structures (selection/if-then-else, multi-way branching/switch or case, and iteration/while-do or repeat-until). Some of these CFG constructs are shown in \reffig{fig:cfg-32}.

\begin{figure}[!ht]
    \includegraphics{images/cfg-figure-3-2.png}
    \caption{if-then-else, switch, and while-do constructs in a CFG.}
    \labfig{fig:cfg-32}
\end{figure}

Other constructs such as repeat-until, nested if, and for loop structures can be constructed in a similar way from these basic constructs.

\begin{example}
\labexample{cfg-exercise}
For the C program below, create a CFG and identify the paths.

\begin{lstlisting}[language=C, caption={A C program that prints the negative integers in an array of 10 integer values.}]
#include <stdio.h>
int main(){
  int array[10] = {-11, 24, 32, -7, 28, 14, 1, -3, -16, 21};
  int k = 0, count = 0;
  printf("Displaying negative integers:\n");
  while (k <= 10){
  	if (array[k] < 0){
  		printf("%d\n", array[k]);
		count++;  	
    }	 
	k++;
  }
  printf("Number of negative integers:%d\n", count);
  return 0;
}
\end{lstlisting}
\end{example}

\begin{marginfigure}[-0.5cm]
    \includegraphics{images/cfg-figure-3-3.png}
    \caption{A CFG for the C program given in \refexample{cfg-exercise}.}
    \labfig{fig:cfg-33}
\end{marginfigure}

A CFG for this program is given in \reffig{fig:cfg-33}.

Some of the paths are identified as follows:
\begin{itemize}
    \item Path1: 1-2-3-4(F)-8-9
    \item Path2: 1-2-3-4(T)-5(T)-6-7-4(F)-8-9
    \item Path3: 1-2-3-4(T)-5(F)-7-4(F)-8-9
    \item Path4: 1-2-3-4(T)-5(T)-6-7-4(T)-5(F)-7-4(F)-8-9
\end{itemize}

\section{McCabe Cyclomatic Complexity}
The concept of cyclomatic complexity metric of a software unit is first defined by McCabe in 1976 \autocite{mccabe1976complexity}. Given a program unit, and a corresponding flow graph (program graph) G with N nodes and E edges, the cyclomatic complexity of the graph $V(G)$ is computed using the formula $V(G) = E - N + 2$. V(G) is always greater than or equal to 1. Each node in the graph $G$ indicates one or more statements (assignment, I/O or control) in the program and the flow of control is represented by directed edges. The basic structures in the McCabe complexity graph are the same as those in the CFG. However, in the McCabe graph, the decision nodes and assignment statement nodes are represented as circled nodes. The CFG given in \reffig{fig:cfg-33} is redrawn in \reffig{fig:cfg-34} using the McCabe graph notation. For simplicity, some of the nodes representing the initialization (assignment) statements are merged into a single node.

\begin{marginfigure}[-1cm]
    \includegraphics{images/cfg-34.png}
    \caption{McCabe program graph for the C program given in \refexample{cfg-exercise}.}
    \labfig{fig:cfg-34}
\end{marginfigure}

Cyclomatic complexity of this graph is simply $V(G) = E - N + 2 = 8 - 7 + 2 = 3$. Three independent paths are shown below:
\begin{itemize}
    \item Path1: 1-2(F)-6-7
    \item Path2: 1-2(T)-3(T)-4-5-2(F)-6-7
    \item Path3: 1-2(T)-3(F)-5-2(F)-6-7
\end{itemize}

The McCabe complexity metric guides the tester about the number of paths to be tested by providing the number of independent paths in the program graph. Independent path is defined as a path that has at least one edge which has not been traversed before in any other paths. In the  software world, it is generally accepted or tried case that this complexity value is less than ten. It is common practice to simplify or subdivide a unit into manageable units with values greater than ten.

\section{Path Selection Criteria}
The control flow graph for a unit can contain multiple paths. It is desirable to create and run a test case for each of these paths. However, in most cases this is costly and time consuming. The test cases prepared by the unit tester should not run the same path more than once. In such a case, resources will be wasted. After all, the path selection must be made according to certain criteria. The recommended criteria for path selection in unit tests are listed below:
\begin{itemize}
        \item All paths: If selected, one can detect almost all the faults. However, for a unit with very large number of paths, this option is not feasible. 
        \item Paths that will meet the full statement coverage requirement: Paths that will meet the full statement coverage requirement: In this option, one aims at executing every statement (assignment, I/O, or decision) at least once in the program unit to achieve 100\% statement coverage (= statementscovered / totalnumberofstatements * 100). This is the weakest coverage criterion in comparison with branch and predicate coverage criteria.
        \item Paths that will meet the full branch coverage requirement: Paths that will meet the complete branch coverage requirement: Branch coverage (=numberofexecutedbranches / totalnumberofbranches * 100) means selecting a path that includes the branch. Complete branch coverage means selecting a number of paths such that every branch is included in at least one path.
        \item Paths that will meet the full predicate coverage: Predicates are expressions that can be evaluated to a Boolean value, i.e., true or false. A predicate may contain Boolean variables,other variables that are compared with the relational operators \{ >, <, =, $\geq$, $\leq$, $\neq$ \}, and Boolean function calls (which returns a true or a false value). In a predicate, logical operators such as, AND, OR, and NOT are used to form logical expressions.A clause is a predicate that does not contain any of the logical operators. For full predicate coverage, test cases need to be constructed so that each predicate ise evaluated to both true and false. 
\end{itemize}

\begin{example}
\labexample{state-branch-coverage}
For the C function \lstinline!isPalindrome! below, create the corresponding McCabe program graph and identify the paths for 100\% statement coverage and complete branch coverage.
\begin{lstlisting}[language=C, caption={A C program to check if a given string is a palindrome or not.}]
#include <stdio.h>
#include <string.h>
 
void isPalindrome(char str[])
{
    // Clean up the sentence from special characters and spaces!
    char cleanstr[200];
    int left, right, i, j = 0;
    for (i = 0; i <= strlen(str); i++){
	   	if ((str[i] >= 'a' && str[i] <= 'z') || (str[i] >= 'A' && str[i] <= 'Z')){
		 	cleanstr[j] = str[i];
		 	j++;
    	}
    }
	// Set up left and right pointers using j
   	left = 0;
    right = j - 1;
 
    // Keep comparing characters while they are same
    while (right > left){
	    if (cleanstr[left++] != cleanstr[right--])
        {
            printf("%s is not a palindrome\n", cleanstr);
            return;
        }
    }
    printf("%s is a palindrome\n", cleanstr);		    
}
// Examples are taken from https://czechtheworld.com/best-palindromes/
int main()
{
    isPalindrome("borrow or rob?");
    isPalindrome("a man, a plan, a canal - panama");
    isPalindrome("everything we see in the world is the creative work of women.");
    isPalindrome("no, it is open on one position.");
    return 0;
}

\end{lstlisting}
\end{example}
A McCabe program graph for \refexample{state-branch-coverage} is given in \reffig{fig:cfg-37}. Five independent paths are given as follows:

\begin{marginfigure}[-8.5cm]
    \includegraphics{images/cfg-37.png}
    \caption{McCabe program graph for the \lstinline!isPalindrome! program.}
    \labfig{fig:cfg-37}
\end{marginfigure}

\begin{itemize}
        \item Path1: 1-2(F)-6-7(F)-8-12
        \item Path2: 1-2(F)-6-7(T)-9(T)-10-12
        \item Path3: 1-2(F)-6-7(T)-9(F)-11-7(F)-8-12
        \item Path4: 1-2(T)-3(F)-5-2(F)-6-7(F)-8-12 
        \item Path5: 1-2(T)-3(T)-4-5-2(F)-6-7(F)-8-12 
         
\end{itemize}
When executed, Path2, Path3, and Path5 provide 100\% statement coverage. Complete branch coverage is possible by adding the execution of Path4.

\begin{example}
\labexample{cfg-prime}
For the C program below, create a CFG and identify the paths for 100\% statement coverage, and complete branch coverage.

\begin{lstlisting}[language=C, caption={A C program to test if a number is prime or not.}]
#include <stdio.h>
int main (){
	int test, i = 2, flag = 0;
	printf("Enter a positive int >=2:");
	scanf("%d", &test);
	while (i <= test / 2){
		if (test % i == 0){
			flag = 1;
			break;
		}
		i++;
	}
	if (test ==  0 || test == 1)
		printf ("%d is neither prime nor composite.\n", test);
	else{
		if(flag == 0)
			printf ("%d is a prime number.\n", test);
		else
			printf ("%d is not a prime number.\n", test);
	}
	return(0);
}
\end{lstlisting}
\end{example}

A CFG for \refexample{cfg-prime} is given in \reffig{fig:cfg-36}. Five independent paths are given as follows:
\begin{itemize}
        \item Path1: 1-2-3(F)-7(T)-12
        \item Path2: 1-2-3(F)-7(F)-9(T)-10-12
        \item Path3: 1-2-3(F)-7(F)-9(F)-11-12
        \item Path4: 1-2-3(T)-4(F)-5-3(F)-7(T)-8-12 
        \item Path5: 1-2-3(T)-4(T)-6-7(T)-8-12 
\end{itemize}

Path2, Path3, Path4, and Path5 provides complete statement and branch coverage. Execution of all these paths provides fFull predicate coverage. 

\begin{figure}[!ht]
    \includegraphics{images/cfg-36.png}
    \caption{CFG for the prime number test program.}
    \labfig{fig:cfg-36}
\end{figure}


\section{Generating Test Cases}

\section{Problems}

\begin{algorithm}
\DontPrintSemicolon
\SetKwInOut{Input}{input}\SetKwInOut{Output}{output}
\Input{$a, b \in \mathbb{Z}^+$}
\Output{GCD of $a$ and $b$.}

\Begin{
    \Repeat{$R = 0$}{
        $R \longleftarrow a \bmod b$\;
        $a \longleftarrow b$\;
        $b \longleftarrow R$\;
    }
    $GCD \longleftarrow b$\;
}

\caption{The pseudo-code of the Euclidean algorithm.}
\label{euclidean-algo}
\end{algorithm}

\begin{enumerate}
    \item Create a CFG for the \lstinline!BinarySearch! function given in \refexample{ex22}. Give a list of possible paths.
    \item Write a C function to sort n integers in descending order using Bubble sort. Draw the McCabe program graph and compute the McCabe cyclomatic complexity for this program and state the number of independent paths accordingly.
    \item Write a C function to find the greatest common divisor (GCD) of two integers M, and N by using the Euclidean algorithm. Draw the McCabe program graph, compute the McCabe cyclomatic complexity for this program and determine the independent paths accordingly.
    The pseudo-code of the Euclidean algorithm is given in Algorithm \ref{euclidean-algo}.
    \item Write the pseudo-code corresponding to the McCabe program graph given in \reffig{fig:cfg-35}. Compute V(G), and determine the independent paths.
    \item For \refexample{cfg-prime}, find the McCabe cyclomatic complexity (number of independent paths) by converting the CFG given in \reffig{fig:cfg-36} to a program graph.
    \item What is an infeasible path in software testing?
    \item Consider Algorithm \ref{infeasible-algo}\sidenote{The algorithm is licensed under a \href{https://creativecommons.org/licenses/by-sa/4.0/}{Creative Commons -Attribution -ShareAlike 4.0 (CC-BY-SA 4.0)}. More information can be found in \href{https://www.educative.io/edpresso/what-is-an-infeasible-path-in-software-testing}{here}.} below consisting of two successive decision structures structures. Is there an infeasible path? If yes, express it in terms of statement labels.
    \item What is
    \item
    \item
\end{enumerate}

\begin{algorithm}
\DontPrintSemicolon
\SetKwInOut{Input}{input}\SetKwInOut{Output}{output}
\Input{$height, a, b, c \in \mathbb{Z}^+$}
\Begin{
    \eIf{$height \leq 26$} {
        $a \longleftarrow 11$\;
    }{
        $b \longleftarrow 13$\;
    }
    \If{$height > 75$}{
        $c \longleftarrow 15$\;
    }
}
\caption{Infeasible path detection.}
\label{infeasible-algo}
\end{algorithm}

\begin{marginfigure}[-21cm]
    \includegraphics{images/cfg-35.png}
    \caption{McCabe program graph for the C program given in \refexample{cfg-exercise}.}
    \labfig{fig:cfg-35}
\end{marginfigure}

    \chapter{Data Flow Testing}
\section{Data Flow Anomaly}
\section{Data Flow Graph (DFG)}
\section{All-paths Data Flow Testing Criteria}
\section{Problems}
\begin{enumerate}
    \item 
    \item 
    \item 
    \item 
    \item 
    \item
    \item 
\end{enumerate}
    \setchapterimage[7.5cm]{images/francesco-ungaro-MJ1Q7hHeGlA-unsplash}
\setchapterpreamble[u]{\margintoc}
\chapter{Integration Testing}
\section{Objectives}
\section{Types of Interfaces and Interface Errors}
\section{Integration Techniques}
\subsection{Incremental Approach}
\subsection{Top-down Approach}
\subsection{Bottom-up approach}
\subsection{Sandwich and Bing-bang Approaches}
\section{Problems}
    \setchapterstyle{kao}
\setchapterpreamble[u]{\margintoc}
\chapter{System Test Categories}
\section{System Test Taxonomy}
\section{Basic Tests}
\section{Functionality Tests}
\section{Robustness Tests}
\section{Interoperability Tests}
\section{Performance Tests}
\section{Stress Tests}
\section{Scalability Tests}
\section{Reliability Tests}
\section{Regression Tests}
\section{Problems}
\begin{enumerate}
    \item 
    \item 
    \item 
    \item 
    \item 
    \item
    \item 
\end{enumerate}
    \setchapterpreamble[u]{\margintoc}
\chapter{Functional Testing}
\section{Howden’s Approach}
\section{Pairwise Testing with Orthogonal Arrays}
\section{Orthogonal Array Generation with DEVELVE}
\section{Equivalence Class Partitioning (ECP)}
\section{Boundary Value Analysis}
\section{Cause-Effect Graphs}
\section{Decision Tables}
\section{Error Guessing}
\section{Problems}
\begin{enumerate}
    \item 
    \item 
    \item 
    \item 
    \item 
    \item
    \item 
\end{enumerate}
    \setchapterpreamble[u]{\margintoc}
\chapter{Software Testing Metrics}
\section{Measurement vs Metrics}
\section{Product Metrics for Testing}
\section{Process Metrics for Testing}
\section{Problems}
\begin{enumerate}
    \item 
    \item 
    \item 
    \item 
    \item 
    \item
    \item 
\end{enumerate}
    \chapter{Acceptance Testing}
\section{Type of Acceptance testing}
\section{Acceptance Criteria}
\section{Test plan}
\section{Problems}
\begin{enumerate}
    \item 
    \item 
    \item 
    \item 
    \item 
    \item
    \item 
\end{enumerate}
    
    \pagelayout{wide} % No margins
    \addpart{Laboratory Studies and Exercises}
    \pagelayout{margin} % Restore margins
    
    \chapter{Introduction}
\label{ch:introduction}
Software testing and test design is very important and inseparable part of the software development process. The application of testing methods is as important as the theory of testing. This manual aims to be a supplementary document to the theoretical lectures of software testing. It can be used as an introductory guide to JUnit Jupiter API or can be followed as a text for laboratory studies. Java is chosen as the primary programming language for the exercises. Each section has two essential parts; the first one introduces the focused subject and the second one presents some exercises on the subject. Sections are designed to cover a 10-weeks semester schedule. However, some sections can be taught as two-week sections.

\section{Brief Introduction to Java Programming Language}
Java is an object-oriented programming language created by Sun engineers James Gosling, Mike Sheridan, and Patrick Naughton in 1991. Java programs are compiled into a special bytecode before being interpreted by Java Virtual Machine (JVM). This bytecode can be thought of as a high-level version of low-level machine languages such as Assembly. JVM interprets the bytecode and makes the system calls and other necessary operations on behalf of the running bytecode. This additional layer sometimes causes Java programs to be a little bit slower than their C/\CC~counterparts. Because of this "compilation then interpretation" stage Java can be classified as a hybrid programming language. 

One advantage of using a JVM is that every operating system has its own implementation of JVM. Therefore, the same Java program can be run in multiple operating systems without any change in the code. This is why Java is known as a platform-independent language.

\subsection{Java Terminology}
Before moving on to more complex topics, let's discuss the most common terminology of Java. Some of the below terms are discussed before and some are new.

\begin{enumerate}
    \item \textbf{Java Virtual Machine (JVM):} A written program is first converted to a low-level language called bytecode by a program called \emph{javac}. Then, JVM interprets the bytecode and makes the operation on behalf of it.
    
    \item \textbf{Bytecode:} As discussed previously, it is a low-level language model to communicate with JVM. It can be physically found in projects as files with \emph{.class} file extension.
    
    \item \textbf{Java Development Kit (JDK):} JDK is a complete development environment to develop Java programs. It contains a compiler, Java Runtime Environment (JRE), Java debuggers, Java documentations, etc.
    
    \item \textbf{Java Runtime Environment (JRE):} JRE is just an environment only capable of running pre-compiled Java programs. One cannot compile Java programs with only JRE installed. JRE includes a browser, JVM, applet supports, and plugins. To be able to run a Java program a computer should have at least JRE installed on it.
    
    \item \textbf{Garbage Collector:} Unlike C/\CC, one cannot delete objects manually in Java. This responsibility is taken care of automatically by a special program called \emph{Garbage Collector}. Garbage collector detects the objects which have not been referenced anymore and deletes them to recollect the memory occupied by them.
    
    \item \textbf{ClassPath:} The classpath is the file path where the Java runtime and Java compiler look for .class files to load. If you want to add external libraries , then you must add them to the classpath.
\end{enumerate}

\subsection{An Overview of a Hello World Program}
Java has many features. Some of them are object-oriented, multithreaded, allows sandbox execution, and the list goes on. Curious readers can find the full list on GeeksForGeeks website\footnote{\url{https://www.geeksforgeeks.org/introduction-to-java/?ref=lbp}}.

The best way to explain some syntax rules is to go through an example. Let's look at the example Java program shown in Listing \ref{lst:java-hello}.

\begin{lstlisting}[caption={Hello world example written in Java.},label=lst:java-hello]
// Demo Java program

// Importing classes from packages
import java.io.*;

// Main class
public class SomeClassName {
    // Main driver method
    public static void main(String[] args) {

        // Print statement
        System.out.println("Hello World!");
    }
}
\end{lstlisting}

The lines starting with \lstinline!//! are called \emph{comment} lines. Compilers ignore the comment lines. Comments can be single line or multiple lines. Multiple line comments start with \verb|/*| and end with \verb|*/|. e.g. \verb|/* A multiline comment */|.

The line \lstinline!import java.io.*;! means that import all the classes of \lstinline!io! package which also belongs to another package called \lstinline!java!. This is generally \textbf{not} a recommended way of importing packages.

Following the import statement, \lstinline!public class SomeClassName! defines a class. Here, \lstinline!public! is an \emph{access modifier}. It defines the places that can access to this class. Then, keyword \lstinline!class! indicates that we are defining a class with the name \lstinline!SomeClassName!. In Java, each file can have only one class (except for the inner classes) and the name of the class and the file that holds the class must be the same. Otherwise, Java compiler raises an error.

Each Java program starts from a \lstinline!static! method called \lstinline|main| which takes a \lstinline|String| array for command-line arguments. This method must be \lstinline|static| because it does not actually belong to the class which defines it and must be called externally by the JVM. In Listing \ref{lst:java-hello}, this method is defined as \lstinline!public static void main(String[] args)!. Here, the access modifier is set to \lstinline|public|. However, it is not necessary. Followed by the \lstinline|static|, the return type of the method is set to \lstinline|void|. This means that the method returns nothing after it is called which makes sense.

In Java, writing and reading operations always work on streams. In this case, by writing \lstinline{System.out.println("Hello world!");} we want to get the \lstinline|System.out| stream which is \lstinline|stdout| pseudo-file (console) and print a newline character terminated string. We can also get the \lstinline|stdin| pseudo-file to be able to read the inputs from the console by utilizing the \lstinline|System.in|.

You are now ready to expand your knowledge about Java further with this basic Java introduction. Some important resources to learn Java are \autocite{schildt2007java,schildt2010java,horstmann_2021}.

\section{Dependency Management with Maven}
Dependency management is a big part of every software development especially if multiple dependencies are involved. Each dependency can also depend on other dependencies and their specific versions. This is a huge problem because sometimes developers need to use an incompatible version as a dependency of the software they are developing which is a dependency of another dependency. These types of problems are hard to solve by humans. Because of this, various helper programs can be used. One of them is Maven, another popular one is Gradle. The list can be long for popular programming languages such as Java.

In this lab, we are going to focus on Maven. Without saying much, one can reach all the detailed information about Maven on Apache Maven Project website\footnote{\url{https://maven.apache.org/guides/getting-started/index.html}}. Let's return to the subject. Each Maven project has the same directory structure shown in Table \ref{tab:maven-dir-layout}.

\begin{table}
    \centering
    \renewcommand{\arraystretch}{1.2}
    \caption{Maven directory layout.}
    \label{tab:maven-dir-layout}
    \begin{tabular}{ll}
        \toprule
        Directory/File & Description \\
        \midrule
        \directory{src/main/java} & Application/Library sources \\
        \directory{src/main/resources} & Application/Library resources \\
        \directory{src/test/java} & Test sources \\
        \directory{src/test/resources} & Test resources \\
        \verb|target| & Output of the build \\
        \verb|pom.xml| & Description of the project \\
        \bottomrule
    \end{tabular}
\end{table}

There can be other directories in the project. A full list can be found in Apache Maven Project website\footnote{\url{https://bit.ly/34VlPvC}}. The \directory{src} directory holds the source code of the software and other resources such as images, database files, etc. The \directory{target} directory contains the output of a build. The most important file for a Maven project is the \lstinline[language={}]|pom.txt| file. This file is an Extensible Markup Language (XML) file to hold all the necessary information about the project and its dependencies. An example \lstinline[language={}]|pom.xml| file is shown in Listing \ref{lst:pom-example}.

\begin{lstlisting}[language=XML,caption={An example pom.xml file.},label=lst:pom-example]
<project xmlns="http://maven.apache.org/POM/4.0.0" xmlns:xsi="http://www.w3.org/2001/XMLSchema-instance" xsi:schemaLocation="http://maven.apache.org/POM/4.0.0 http://maven.apache.org/xsd/maven-4.0.0.xsd">
    <modelVersion>4.0.0</modelVersion>
 
    <groupId>com.mycompany.app</groupId>
    <artifactId>my-app</artifactId>
    <version>1</version>
    
    <properties>
        <mavenVersion>3.0</mavenVersion>
    </properties>
 
    <dependencies>
        <dependency>
            <groupId>org.apache.maven</groupId>
            <artifactId>maven-artifact</artifactId>
            <version>${mavenVersion}</version>
        </dependency>
        <dependency>
            <groupId>org.apache.maven</groupId>
            <artifactId>maven-core</artifactId>
            <version>${mavenVersion}</version>
        </dependency>
    </dependencies>
</project>
\end{lstlisting}

Each \lstinline[language={}]|pom.xml| file is actually an XML Schema Definition (XSD). At the top level, a project element defines the namespace information in its attributes. In the subelements, the version of the project, group id, artifact id, and properties are declared. Following the properties, a complex element called \lstinline[language={XML}]|<dependencies>| declares the dependencies of the project. There is a wide variety of elements to use inside the project. Full list can be obtained from here\footnote{\url{https://maven.apache.org/pom.html}}.

\section{Installing Eclipse and Setting Up a Maven Project}
Fortunately, most of the time we do not have to deal with constructing the directory structure or writing the \lstinline[language={}]|pom.xml| file. Most Integrated Development Environments (IDE) prepare those structures with project creation wizards and automate the building tasks. You can write Java code even in the simple Notepad program. However, using an IDE greatly reduces the overhead of building and writing software. There are many great choices when it comes to IDEs. In this manual, Eclipse IDE is chosen. To install Eclipse in Windows, just go to the official website\footnote{\url{https://www.eclipse.org/downloads/}} and download the download manager and install Eclipse for Java. For Linux, just use the package manager of the distribution that you are using. Notice that, it is assumed that either OpenJDK or Oracle JDK has been already installed in your computer. Alternatively, you can install Eclipse independently from the distribution via Snap\footnote{\url{https://snapcraft.io/store}} in Linux or via Chocolatey\footnote{\url{https://community.chocolatey.org/packages}} in Windows 10 or above.

After installing Eclipse, a standard welcome screen should be opened as shown in Figure \ref{fig:eclipse-welcome}. In this screen, close the welcome tab and go to \menu{File > New > Project...}.

\begin{figure}[H]
    \centering
    \includegraphics[width=0.95\textwidth]{images/eclipse-welcome.png}
    \caption{Welcome screen of Eclipse.}
    \label{fig:eclipse-welcome}
\end{figure}

A new dialog window should be showing up as in Figure \ref{fig:eclipse-project}. Search and choose the \emph{Maven Project}. After that, a wizard (shown in Figure \ref{fig:maven-new}) asks you a few questions about how you want to configure your new project.

\begin{figure}[H]
    \centering
    \includegraphics[width=\textwidth]{images/eclipse-project.png}
    \caption{Create a new project window.}
    \label{fig:eclipse-project}
\end{figure}

In Figure \ref{fig:maven-new}, make sure that \emph{Create a simple project} checkbox is checked. This will allow us to skip some unnecessary configuration options in the next pages. Click \keys{Next >} and you should see the dialog window shown in Figure \ref{fig:maven-app-conf}.

\begin{figure}[H]
    \centering
    \includegraphics[width=\textwidth]{images/maven-new.png}
    \caption{Maven project wizard landing page.}
    \label{fig:maven-new}
\end{figure}

In Figure \ref{fig:maven-app-conf}, there are only two fields that must be filled before clicking to the \keys{Finish} button. The first one is the \emph{Group Id} field. Here, you should write a general package name that normally should hold all of your similar projects. For example, since we are going to write a project specifically for the SE344 lab that is offered at Atılım University, we can write something like this: \lstinline|edu.atilim.se344|. The first part, \lstinline|edu|, indicates the top-level domain as on the Internet. Then, it is followed by the company name; in this case, it is our university. Finally, followed by the name of the course. In \emph{Artifact Id}, we just give a name to our build target/executable. In this case, it is \lstinline|lab1|. When we build the project, our build target will be named like \lstinline[language={}]|lab1-0.0.1-SNAPSHOT.jar|. When you have finished filling the necessary fields, click \keys{Finish} button. The new project should be created.

\begin{figure}[H]
    \centering
    \includegraphics[width=\textwidth]{images/maven-app-conf.png}
    \caption{Maven project wizard final page.}
    \label{fig:maven-app-conf}
\end{figure}

Before moving on to the next topic, there is a small problem that we need to address. In some versions of Eclipse, Java projects have a default Java version of 1.5. This is a problem for us. Therefore, right-click to the newly created project and click \menu{Properties}. The dialog window shown in Figure \ref{fig:java-version} should be opened. Here, click \emph{Java Compiler} in the list view. Untick the checkbox starting \emph{Use compliance from execution...} and select 1.8 from \emph{Compiler compliance level} checkbox. Click \keys{Apply \& Close} button when you have finished with the project properties.

\begin{figure}[H]
    \centering
    \includegraphics[width=\textwidth]{images/java-version.png}
    \caption{Java compiler properties for a Java project.}
    \label{fig:java-version}
\end{figure}

This concludes our project creation step. Now, we need to add necessary dependencies to our project's \lstinline[language={}]|pom.xml| file. In this case, there is only one dependency\todo{There are multiple dependencies.} and it is JUnit Jupiter API. In the next section, we are going to see how to add such dependencies to our project. Also, we need to make sure that our Maven build system is compatible with JUnit. We will add a plugin to our project to comply with JUnit.

\section{Introduction to JUnit Jupiter API}
JUnit is one of the leading unit testing frameworks for Java. It has a standard API that supplies all the necessary methods and classes for a complete test suite. In fact, it influences many other unit test frameworks in other programming languages. In Java, libraries are added to the projects by adding related \lstinline[language={}]|.jar| files to the classpath and importing them into the program. The responsibility of managing those libraries completely belongs to the developer. That is a challenging problem as stated in a previous section. Therefore, using a dependency manager is always a good idea.

In this section, we will add JUnit Jupiter API (a.k.a. JUnit 5) to our previously created project and choose a Maven plugin version that works with the JUnit API. Remember from the previous \verb|pom.xml| examples, each dependency that we want to add to our project must go to a complex element called \lstinline[language=XML]|<dependencies>|. We will use JUnit Jupiter API version 5.8.2 in this example. Go to the Maven repository website, search for the package \emph{junit}, and click the version \lstinline[language={}]|5.8.2|\footnote{\url{https://mvnrepository.com/artifact/org.junit.jupiter/junit-jupiter-api/5.8.2}}. In the middle of the page, you should see a code snippet as shown in Listing \ref{lst:pom-junit}. Copy the code snippet and paste it inside the \lstinline[language=XML]|<dependencies>| element.\todo{JUnit Engine dependency is also needed to run the test cases with BeforeAll and AfterAll annotations.}

\begin{lstlisting}[language=XML,caption={JUnit Jupiter API version 5.8.2 dependency element.},label=lst:pom-junit]
<dependency>
    <groupId>org.junit.jupiter</groupId>
    <artifactId>junit-jupiter-api</artifactId>
    <version>5.8.2</version>
    <scope>test</scope>
</dependency>
\end{lstlisting}

After adding JUnit Jupiter API, we need to indicate a specific version of Maven Compiler Plugin\todo{This part needs to be revised.} that works with the newest JUnit Jupiter API. In this case, it is the version \lstinline[language={}]|3.8.1|. Add the code snippet\footnote{This plugin may not be needed for the recent versions of Eclipse.} shown in Listing\todo{This one is the wrong snippet and must be substituted with the surefire plugin.} \ref{lst:pom-maven-plugin} to your \lstinline[language={}]|pom.xml| file under the top-level \lstinline[language=XML]|<project>| element.

\begin{lstlisting}[language=XML,caption={A compatible Maven Compiler Plugin JUnit Jupiter API v5.8.2..},label=lst:pom-maven-plugin]
<build>
    <pluginManagement>
        <plugins>
            <plugin>
                <groupId>org.apache.maven.plugins</groupId>
                <artifactId>maven-compiler-plugin</artifactId>
                <version>3.8.1</version>
            </plugin>
        </plugins>
    </pluginManagement>
</build>
\end{lstlisting}

In the end, your \verb|pom.xml| file should look like Listing \ref{lst:pom-future}.

\begin{minipage}{\textwidth}
    \begin{lstlisting}[language=XML,caption={pom.xml file for future exercises.},label=lst:pom-future]
<project xmlns="http://maven.apache.org/POM/4.0.0"
    xmlns:xsi="http://www.w3.org/2001/XMLSchema-instance"
    xsi:schemaLocation="http://maven.apache.org/POM/4.0.0 https://maven.apache.org/xsd/maven-4.0.0.xsd">
    <modelVersion>4.0.0</modelVersion>
    <groupId>edu.atilim.se344</groupId>
    <artifactId>lab1</artifactId>
    <version>0.0.1-SNAPSHOT</version>
    
    <!-- Set the compiler version to Java 1.8 for compatibility with JUnit. -->
    <properties>
        <maven.compiler.source>1.8</maven.compiler.source>
        <maven.compiler.target>1.8</maven.compiler.target>
    </properties>

    <dependencies>
        <dependency>
            <groupId>org.junit.jupiter</groupId>
            <artifactId>junit-jupiter-api</artifactId>
            <version>5.8.2</version>
            <scope>test</scope>
        </dependency>
    </dependencies>

    <!-- This part can be omitted in recent versions of Eclipse. -->
    <build>
        <pluginManagement>
            <plugins>
                <plugin>
                    <groupId>org.apache.maven.plugins</groupId>
                    <artifactId>maven-compiler-plugin</artifactId>
                    <version>3.8.1</version>
                </plugin>
            </plugins>
        </pluginManagement>
    </build>
</project>
    \end{lstlisting}
\end{minipage}
    \chapter{Unit Testing w/ JUnit Jupiter API}
JUnit Jupiter API has a very stable and consistent API. There are two essential resources to learn it. The first one is the official JUnit 5 User Guide\footnote{\url{https://junit.org/junit5/docs/current/user-guide/}} and the second one is the official JUnit 5 Java Docs\footnote{\url{https://junit.org/junit5/docs/current/api/}}. In this section, the important assert types of the JUnit 5 is introduced without any meaningful context or formal methods.

\section{Writing Tests}
Let's assume that we are developing a \emph{Calculator} program and we are using the Test-Driven Development approach. We want to test the addition functionality of the calculator. In JUnit, our test case would look like Listing \ref{lst:calc-add-test}. In the code, a class called \verb|Calculator| is assumed to exist in the package \lstinline|edu.atilim.se344|. One of its methods is \lstinline!add(int a, int b);! and we are testing this method.

To create a test case, first, we need to create a \lstinline[language={}]|.java| file inside \directory{src/test/java}. To do that, right-click to the \directory{src/test/java} and click \menu{New>Class} from the context menu. The name of the file is not important as long as the same with the name of the class inside. Inside of the file, we need a class. The contents of the class need to comply with some rules.

\begin{lstlisting}[caption={A test case for testing the addition functionality of the Calculator class.},label=lst:calc-add-test]
import static org.junit.jupiter.api.Assertions.assertEquals;

import edu.atilim.se344.Calculator;
import org.junit.jupiter.api.Test;

class CalculatorTests {

    private final Calculator calculator = new Calculator();

    @Test
    void addition() {
        assertEquals(2, calculator.add(1, 1));
    }
}
\end{lstlisting}

The test class cannot be \lstinline{abstract} and must have at least one method that is annotated with \lstinline!@Test!. In our example, \lstinline|CalculatorTests| class have a constant Calculator instance \lstinline|calculator| and a \lstinline!void addition();! method that is annotated by \lstinline!@Test! which causes the method to be automatically called by the JUnit framework. There are lots of annotations that help both JUnit to identify the role of the method and the developer to troubleshoot any problem.

A more complete and standard test class should have some other special methods. Such as a method that is called before any of the other methods, a method that is called before each test case, test cases themselves, a method that is called after each test case, and a method that is called after every test case is called. Such a standard test class is proposed by the official JUnit guide. You can see it in Listing \ref{lst:std-test-class}.

\begin{lstlisting}[caption={A standard test class.},label=lst:std-test-class]
import static org.junit.jupiter.api.Assertions.fail;
import static org.junit.jupiter.api.Assumptions.assumeTrue;

import org.junit.jupiter.api.AfterAll;
import org.junit.jupiter.api.AfterEach;
import org.junit.jupiter.api.BeforeAll;
import org.junit.jupiter.api.BeforeEach;
import org.junit.jupiter.api.Disabled;
import org.junit.jupiter.api.Test;

class StandardTests {

    @BeforeAll
    static void initAll() {
    }
    
    @BeforeEach
    void init() {
    }
    
    @Test
    void succeedingTest() {
    }
    
    @Test
    void failingTest() {
        fail("a failing test");
    }
    
    @Test
    @Disabled("for demonstration purposes")
    void skippedTest() {
        // not executed
    }
    
    @Test
    void abortedTest() {
        assumeTrue("abc".contains("Z"));
        fail("test should have been aborted");
    }
    
    @AfterEach
    void tearDown() {
    }
    
    @AfterAll
    static void tearDownAll() {
    }
}
\end{lstlisting}

The developer can also assign more readable names to her/his test cases with the \lstinline!@DisplayName! annotation. It takes a string e.g. \lstinline!@DisplayName("Some interesting test case")!. It can take all valid Unicode encoded strings even emojis.

\section{Assert Types}
There are many assert types that can be used in various situations. The full list can be found in the documentation\footnote{\url{https://junit.org/junit5/docs/current/api/org.junit.jupiter.api/org/junit/jupiter/api/Assertions.html}}. Most of them are different overloads of the same assert function. The widely used ones are given in Table \ref{tab:junit-asserts}.

\begin{table}[H]
    \centering
    \renewcommand{\arraystretch}{1.2}
    \caption{Some of the assert methods that are defined in JUnit API.}
    \label{tab:junit-asserts}
    \begin{adjustbox}{max width=\textwidth}
        \begin{tabular}{ll}
            \toprule
            Method Prototype & Description \\
            \midrule
            \lstinline!assertAll(String, Collection<Executable>)! & Assert that all supplied executables do not throw exceptions.\\
            \lstinline!assertArrayEquals(boolean[] expected, boolean[] actual)! & Assert that expected and actual boolean arrays are equal.\\
            \lstinline!assertEquals(byte expected, byte actual)! & Assert that expected and actual are equal.\\
            \lstinline!assertFalse(boolean condition)! & Assert that the supplied condition is false.\\
            \lstinline!assertNotEquals(byte unexpected, byte actual)! & Assert that expected and actual are not equal.\\
            \lstinline!assertTrue(boolean condition)! & Assert that the supplied condition is true.\\
            \bottomrule
        \end{tabular}
    \end{adjustbox}
\end{table}

Notice that there are a huge amount of overloads of these methods. Since our main objective is not presenting the whole list, only a very small amount of them are projected into the Table \ref{tab:junit-asserts}.

\section{Running Tests}
Running test in Eclipse is fairly straightforward. You can right click to the current project in the \emph{Package Explorer}. In the context menu, follow the menu option \menu{Run As > Maven test}. This will trigger the automatic build process of Maven and runs all the tests inside the \directory{src/test/java} path.

A successful test run should produce an output like this:

\begin{lstlisting}[language={},caption={A log output from a Maven test run.}]
[INFO] Scanning for projects...
[INFO] 
[INFO] -----------------------< edu.atilim.se344:lab1 >------------------------
[INFO] Building lab1 0.0.1-SNAPSHOT
[INFO] --------------------------------[ jar ]---------------------------------
[INFO] 
[INFO] --- maven-resources-plugin:2.6:resources (default-resources) @ lab1 ---
[WARNING] Using platform encoding (UTF-8 actually) to copy filtered resources, i.e. build is platform dependent!
[INFO] Copying 0 resource
[INFO] 
[INFO] --- maven-compiler-plugin:3.8.1:compile (default-compile) @ lab1 ---
[INFO] Nothing to compile - all classes are up to date
[INFO] 
[INFO] --- maven-resources-plugin:2.6:testResources (default-testResources) @ lab1 ---
[WARNING] Using platform encoding (UTF-8 actually) to copy filtered resources, i.e. build is platform dependent!
[INFO] Copying 0 resource
[INFO] 
[INFO] --- maven-compiler-plugin:3.8.1:testCompile (default-testCompile) @ lab1 ---
[INFO] Nothing to compile - all classes are up to date
[INFO] 
[INFO] --- maven-surefire-plugin:2.12.4:test (default-test) @ lab1 ---
[INFO] Surefire report directory: /home/tustunkok/eclipse-workspace/lab1/target/surefire-reports

-------------------------------------------------------
 T E S T S
-------------------------------------------------------
Running lab1.CalculatorTest
Tests run: 1, Failures: 0, Errors: 0, Skipped: 0, Time elapsed: 0.002 sec

Results :

Tests run: 1, Failures: 0, Errors: 0, Skipped: 0

[INFO] ------------------------------------------------------------------------
[INFO] BUILD SUCCESS
[INFO] ------------------------------------------------------------------------
[INFO] Total time:  1.218 s
[INFO] Finished at: 2022-02-03T15:54:36+03:00
[INFO] ------------------------------------------------------------------------
\end{lstlisting}

\section{Exercises}
Assume that the following calculator class is given to you. You are responsible for writing the necessary test cases for the given class. Remember that there might be mistakes in the given implementations throughout all exercises in the manual. Do not take them as exact true cases.

\begin{lstlisting}[caption={A Calculator class implementation in Java.},label=lst:java-calc]
package edu.atilim.se344;

public class Calculator {
    public int add(int n1, int n2) {
        return n1 + n2;
    }
    
    public int sub(int n1, int n2) {
        return n1 - n2;
    }
    
    public int div(int n1, int n2) {
        return n1 / n2;
    }
    
    public int mul(int n1, int n2) {
        return n1 * n2;
    }
}
\end{lstlisting}

\begin{exercise}
    Write the individual test cases for each of the methods of the Calculator class.
\end{exercise}

\begin{solution}
    \begin{lstlisting}[caption={Trivial unit tests for the Calculator class.}]
import static org.junit.jupiter.api.Assertions.assertEquals;

import edu.atilim.se344.Calculator;
import org.junit.jupiter.api.Test;

class CalculatorTests {

    private final Calculator calculator = new Calculator();

    @Test
    void testAddition() {
        assertEquals(6, calculator.add(1, 5));
    }
    
    @Test
    void testSubtraction() {
        assertEquals(-5, calculator.sub(2, 7));
    }
    
    @Test
    void testDivision() {
        assertEquals(0, calculator.div(5, 8));
    }
    
    @Test
    void testMultiplication() {
        assertEquals(16, calculator.mul(8, 2));
    }
}
    \end{lstlisting}
\end{solution}

\begin{exercise}
    Suppose you are going to extend the functionality of the Calculator class by adding mean calculation for a list. You are utilizing Test-Driven Development (TDD) technique to write the feature. Add the new feature to the class step by step.
\end{exercise}

\begin{solution}
    First, we have to write the test to see it fails.
    \begin{lstlisting}[caption={A unit test to testing the mean method of the Calculator class.}]
import static org.junit.jupiter.api.Assertions.assertEquals;

import edu.atilim.se344.Calculator;
import org.junit.jupiter.api.Test;

class CalculatorTests {

    private final Calculator calculator = new Calculator();

    ...
    
    @Test
    void testMean() {
        assertEquals(4.5f, calculator.mean(2, 3, 4));
    }
}
    \end{lstlisting}
    Then, add the simplest implementation that should pass the test.
    \begin{lstlisting}[caption={The simplest implementation to pass the test.}]
package edu.atilim.se344;

public class Calculator {
    
    ...
    
    public float mean(int... numbers) {
        return 4.5f;
    }
}
    \end{lstlisting}
    Write another test that will be failed by the previous implementation.
    \begin{lstlisting}[caption={Another test that invalidates the previous implementation.}]
import static org.junit.jupiter.api.Assertions.assertEquals;

import edu.atilim.se344.Calculator;
import org.junit.jupiter.api.Test;

class CalculatorTests {

    private final Calculator calculator = new Calculator();

    ...
    
    @Test
    void testMean() {
        assertEquals(12.0f, calculator.mean(10, 20, 8, 4, 18));
    }
}
    \end{lstlisting}
    Finally, write the actual implementation that you think correct and run the tests again to see that all of them are passed.
    \begin{lstlisting}[caption={A correct implementation of the mean operation.}]
package edu.atilim.se344;

public class Calculator {
    
    ...
    
    public float mean(int... numbers) {
        float sum = 0.0f;
        for (int i : numbers) {
            sum += i;
        }
        return sum / numbers.length;
    }
}
    \end{lstlisting}
    This is a very long way of writing code with TDD. Normally, the first steps are not taken into account and directly passed through the correct implementation part after writing the first test.
\end{solution}

\begin{exercise}
    Suppose that we are adding multiplication operation to our Calculator implementation. However, we are going to implement it as a repeated addition operation as in a primitive microcontroller architecture. Our implementation must only work on floating-point numbers and not integers.
    
    \begin{lstlisting}[caption={An implementation for multiplication operation with a loop.}]
package edu.atilim.se344;

public class Calculator {
    
    ...
    
    public double crudeMultiplication(double num1, double num2) {
        double result = 0.0;
        for (int i = 0; i < num2; i++) {
            result += num1;
        }
        
        return result;
    }
}
    \end{lstlisting}
    
    Write the necessary test case to test such an implementation. Decrease and increase the precision value to see the test case is failed for smaller values.
\end{exercise}

\begin{solution}
        \begin{lstlisting}[caption={A floating-point assert statement with precision.}]
import static org.junit.jupiter.api.Assertions.assertEquals;

import edu.atilim.se344.Calculator;
import org.junit.jupiter.api.Test;

class CalculatorTests {

    private final Calculator calculator = new Calculator();

    ...
    
    @Test
    @DisplayName("Operations with floating-point numbers are dangerous!")
    void testMean() {
        assertEquals(10000.0, calculator.crudeMultiplication(0.1, 100000, 1e-8));
    }
}
    \end{lstlisting}
\end{solution}
    \input{labs/03-mccabes-cyclomatic-complexity}
    \chapter{Data Flow Testing}
\section{Data Flow Anomaly}
\section{Data Flow Graph (DFG)}
\section{All-paths Data Flow Testing Criteria}
\section{Problems}
\begin{enumerate}
    \item 
    \item 
    \item 
    \item 
    \item 
    \item
    \item 
\end{enumerate}
    \chapter{Static Unit Testing}
The main goal of static unit testing is to find the defects as close as to their point of origin. The review techniques that are applied in the static unit testing are \emph{inspection} and \emph{walkthrough} \autocite{naik2011software}.
\begin{itemize}
    \item \textbf{Inspection:} It is a step-by-step peer group review of a work product, with each step checked against predetermined criteria.
    \item \textbf{Walkthrough:} It is a review where the author leads the team through a manual or simulated execution of the product using predefined scenarios.
\end{itemize}
In this section, only inspection will be considered. A given code will be examined against a checklist by multiple groups of students.

\section{Code Review}
Code quality is a crucial concept for many organizations. Code review is a technique that aims to increase code quality. It can be performed by either an expert or a computer. Nowadays, almost all compilers have a static code analyzer that performs some of the code review tasks. Generally, a checklist is utilized to perform a code review. This checklist can be language-specific or general. A general checklist is given in the list below.

Organizations can develop their own checklists. One important point while doing so is that if a language-specific checklist is produced, then a new checklist should be produced for each language that is used.

After performing an inspection, a report is created which is signed by all participants. The report includes the found defects and problems, the degree of importance of the found problems, and the judgments of participants. The report has three acceptance criteria. They are \emph{accept}, \emph{conditional accept}, and \emph{reinspect}.

\begin{enumerate}[nosep]
    \item Does the code do what has been specified in the design specification?
    \item Does the procedure used in the module solve the problem correctly?
    \item Does a software module duplicate another existing module which could be reused?
    \item If library modules are being used, are the right libraries and the right versions of the libraries being used?
    \item Does each module have a single entry point and a single exit point? Multiple exit and entry point programs are harder to test.
    \item Is the cyclomatic complexity of the module more than 10? If yes, then it is extremely difficult to adequately test the module.
    \item Can each atomic function be reviewed and understood in 10–15 minutes? If not, it is considered to be too complex.
    \item Have naming conventions been followed for all identifiers, such as pointers, indices, variables, arrays, and constants? It is important to adhere to coding standards to ease the introduction of a new contributor (programmer) to the development of a system.
    \item Has the code been adequately commented upon?
    \item Have all the variables and constants been correctly initialized? Have correct types and scopes been checked?
    \item Are the global or shared variables, if there are any, carefully controlled?
    \item Are there data values hard coded in the program? Rather, these should be declared as variables.
    \item Are the pointers being used correctly?
    \item Are the dynamically acquired memory blocks deallocated after use?
    \item Does the module terminate abnormally? Will the module eventually terminate?
    \item Is there a possibility of an infinite loop, a loop that never executes, or a loop with a premature exit?
    \item Have all the files been opened for use and closed at termination?
    \item Are there computations using variables with inconsistent data types? Is overflow or underflow a possibility?
    \item Are error codes and condition messages produced by accessing a common table of messages? Each error code should have a meaning, and all of the meanings should be available at one place in a table rather than scattered all over the program code.
    \item Is the code portable? The source code is likely to execute on multiple processor architectures and on different operating systems over its lifetime. It must be implemented in a manner that does not preclude this kind of a variety of execution environments.
    \item Is the code efficient? In general, clarity, readability, or correctness should not be sacrificed for efficiency. Code review is intended to detect implementation choices that have adverse effects on system performance.
\end{enumerate}

\section{Exercises}
Given the following problem;
\begin{displayquote}
    Write a function called \lstinline!bool multiple(int, int)! that determines whether the second integer is a multiple of the first for a pair of integers. The function should take two integer arguments and return true if the second is a multiple of the first, otherwise false. Use this function in a program that inputs a series of pair of integers.
\end{displayquote}
a solution is proposed. It might be a correct implementation or not. It is not important for the context of this section.
\begin{lstlisting}[language=C++,caption={A \CC~program that confirms a number is multiple of another.}]
#include <iostream>
using namespace std;

bool multiple(int, int);
int func2(int);

int main() {
    int x, num2;
    bool a;
    cout << "Enter two integers: ";
    cin >> x >> num2;
    a=multiple(x, num2);
    if(a) {
        cout << num2 << " is a multiple of " << x;
    }
    else
        cout << num2 << " is not a multiple of " << x;
}

bool multiple(int X, int num2) {
    if (num2 % X = 0)
        return true;
        else return false;
}

bool func2(int n) {
    int i;
    for(i = 2; i <= n / 2; i++) {
        if(n % i== 0)
            return false;
    }
    return true;
}
\end{lstlisting}
Perform the following exercises.

\begin{exercise}
    Perform a code review to check problems/defect types in this C Program considering the list in the previous section. Write the line number and the type of the problem/defect to the following table.
    
    \begin{table}[H]
    \centering
    \renewcommand{\arraystretch}{1.2}
    \caption{List of found defects.}
    \label{tab:defects-ex}
    \begin{tabular}{p{0.05\textwidth}|p{0.3\textwidth}|p{0.55\textwidth}}
        \toprule
        \# & Line number & Type of the problem/defect\\
        \midrule
        1 & & \\
        \midrule
        2 & & \\
        \midrule
        3 & & \\
        \midrule
        4 & & \\
        \midrule
        5 & & \\
        \midrule
        6 & & \\
        \midrule
        7 & & \\
        \midrule
        8 & & \\
        \midrule
        9 & & \\
        \midrule
        10 & & \\
        \bottomrule
    \end{tabular}
\end{table}
\end{exercise}

\begin{solution}
    No solution has been suggested.
\end{solution}

\begin{exercise}
    Read the following description of the process of the passenger check of a Passenger Service System. Perform a review to check whether all the steps are met in terms of activities in the following activity diagram. Complete the diagram by adding the missing parts.
    
    \begin{displayquote}
        In a Passenger Service System, when a passenger arrives at the airport to check in, (s)he first shows his or her ticket at the check-in counter. The ticket will be checked for its validity. If the ticket is not OK, the passenger will be referred to customer service. If the ticket is OK, the passenger will check his or her luggage. If the luggage has excess weight he or she will pay an additional fee. The luggage will be forwarded to baggage transportation. The passenger receives his or her boarding pass. Note that this activity is between the passenger and the passenger service. Another solution is to make the system visible to show the interaction between the passenger service and the system.
    \end{displayquote}
    
    \begin{figure}[H]
        \centering
        \includegraphics[width=0.55\textwidth]{images/passenger-activity.jpg}
        \caption{Passenger Service System passenger checks in activity diagram.}
        \label{fig:pass-checks-in}
    \end{figure}
\end{exercise}

\begin{solution}
    No solution has been suggested.
\end{solution}

    \chapter{Black-Box Testing Strategies}
Black box testing deals with testing inputs and outputs considering a given requirement. The problem in black box testing is the size of the input space. The number of test inputs can quickly reach infeasible sizes \autocite{burnstein2006practical}. For example, testing a calculator program's addition functionality can be made with an infinite number of test inputs, i.e. negative integers, positive integers, negative floating-point numbers, positive floating-point numbers, complex numbers and its versions, etc. Nobody has such resources or time. Because of that, there are different testing techniques that can be applied to different testing scenarios. In this chapter, some of them and their applications are presented through exercises.

\section{Equivalence Class Partitioning}
The problems with the input domain of a software-under-test can be resolved by a method called Equivalence Class Partitioning (ECP). When you look at the software-under-test as a black box, the input domain can be partitioned into several classes. The members of each class are assumed to cause the same effect on the software-under-test. When a defect is detected with input from a specific class, all of the other elements from that class are assumed to cause the same defect and vice versa. This is a limitation of ECP.

The classes from ECP are chosen in several different ways. The tester analyzes the requirements for \emph{interesting} input conditions and partitions the input domain accordingly. Then, develops test cases from these partitions. A real-life example mirrors the test case generation techniques from white box testing techniques but instead of using the source code, the requirements are utilized for branch discovery. For example, from the textbook \autocite{burnstein2006practical} the specifications and the relevant equivalence classes are given below:

\begin{algorithm}
\caption{Square root specification \autocite{burnstein2006practical}.}
\label{alg:square-root}
    \begin{algorithmic}[1]
        \Require{$x, y \in \mathds{R}$}
        \Function{squareroot}{$x$}
            \If{$x \ge 0.0$}
                \State\Call{message}{$x$}
            \EndIf
            \If{$y \ge 0.0$ and $y * y \approx x$}
                \State\Call{reply}{$y$}
            \Else
                \Raise{Imaginary square root exception}
            \EndIf
            \State\Return{$y$}
        \EndFunction
    \end{algorithmic}
\end{algorithm}

\begin{enumerate}[noitemsep]
    \item[EC1.] The input variable x is real, valid.
    \item[EC2.] The input variable x is not real, invalid.
    \item[EC3.] The value of x is greater than 0.0, valid.
    \item[EC4.] The value of x is less than 0.0, invalid.
\end{enumerate}

\section{Boundary Value Analysis}
Boundary Value Analysis (BVA) is a nice addition to strengthen the ECP. Most defects generally occur at the class boundaries. These boundaries are valuable to find the defects. In ECP, any input value from a class can be used to test the black box. On the other hand, BVA requires testing of software-under-test with boundary values of equivalence classes. A tester should create test cases with valid inputs from the edges of the equivalence classes in addition to the test cases with invalid inputs. Suppose that the following specifications \autocite{burnstein2006practical} are given to you.

\begin{displayquote}
    The input specification for the module states that a widget identifier should consist of 3–15 alphanumeric characters of which the first two must be letters. We have three separate conditions that apply to the input: (i) it must consist of alphanumeric characters, (ii) the range for the total number of characters is between 3 and 15, and, (iii) the first two characters must be letters.
\end{displayquote}

First, the tester should determine the equivalence classes:
\begin{enumerate}[noitemsep]
    \item[EC1.] Part name is alphanumeric, valid.
    \item[EC2.] Part name is not alphanumeric, invalid.
    \item[EC3.] The widget identifier has between 3 and 15 characters, valid.
    \item[EC4.] The widget identifier has less than 3 characters, invalid.
    \item[EC5.] The widget identifier has greater than 15 characters, invalid.
    \item[EC6.] The first 2 characters are letters, valid.
    \item[EC7.] The first 2 characters are not letters, invalid.
\end{enumerate}

After determining the ECs, the classes are split into invalid and valid as shown in Table \ref{tab:ec-reporting}.

\begin{table}[H]
    \centering
    \renewcommand{\arraystretch}{1.2}
    \caption{Equivalence class reporting table.}
    \label{tab:ec-reporting}
    \begin{adjustbox}{max width=\textwidth}
        \begin{tabular}{lll}
            \toprule
            \thead{Condition} & \thead{Valid Equivalence Classes} & \thead{Invalid Equivalence Classes}\\
            \midrule
            1 & EC1 & EC2\\
            2 & EC3 & EC4, EC5\\
            3 & EC6 & EC7\\
            \bottomrule
        \end{tabular}
    \end{adjustbox}
\end{table}

Now, the tester can derive specific test cases from boundaries of both valid and invalid ECs. For example: 

\begin{table}[H]
    \centering
    \renewcommand{\arraystretch}{1.2}
    \caption{Summary of test inputs using equivalence class partitioning and boundary value analysis for sample module. Table taken from \autocite{burnstein2006practical}.}
    \label{tab:ec-summary}
    \begin{adjustbox}{max width=\textwidth}
        \begin{threeparttable}
            \begin{tabular}{llll}
                \toprule
                \thead{Test Case ID} & \thead{Input Values} & \thead{\makecell{Valid ECs and\\Bounds Covered}} & \thead{\makecell{Invalid ECs and\\Bounds Covered}}\\
                \midrule
                1 & abc1 & EC1, EC3 (ALB\tnote{1} ), EC6 & \\
                2 & ab1 & EC1, EC3 (LB\tnote{2} ), EC6 & \\
                3 & abcdef123456789 & EC1, EC3 (UB\tnote{3} ), EC6 & \\
                4 & abcde123456789 & EC1, EC3 (BUB\tnote{4} ), EC6 & \\
                5 & abc* & EC3 (ALB), EC6 & EC2\\
                6 & ab & EC1, EC6 & EC4 (BLB\tnote{5} )\\
                7 & abcdefg123456789 & EC1, EC6 & EC5 (AUB\tnote{6} )\\
                8 & a123 & EC1, EC3 (ALB) & EC7\\
                9 & abcdef123 & EC1, EC3, EC6 & \\
                \bottomrule
            \end{tabular}
            \begin{tablenotes}
                \item[1] a value just above the lower boundary
                \item[2] the value on the lower boundary
                \item[3] the value on the upper bound
                \item[4] a value just below the upper bound
                \item[5] a value just below the lower bound
                \item[6] a value just above the upper bound
            \end{tablenotes}
        \end{threeparttable}
    \end{adjustbox}
\end{table}

After the determination of expected outputs, the logs of the test cases are recorded. The actual outputs of the tests are compared with the expected outputs to decide the fail/pass status of the tests. In addition to the boundary values, a midpoint from ECs should also be included in the test cases as a typical case. Although BVA suggests a more specific zone to choose input values than ECP testing, these input values are merely non-unique. A tester can choose many different test input values.

\section{Cause-and-Effect Graphing}
Combining multiple conditions in EC cannot be performed intentionally. Some test cases may permit combining conditions by nature and some do not. The cause-and-effect graphing technique is developed to express causes and their effects in a graphical language. The visualization of causes and their effects greatly helps the tester to combine conditions to disclose inconsistencies that normally might not show up.

To produce a cause-and-effect graph, the tester must transform the specification to a graph that resembles a digital logic circuit. The process then starts with the decomposition of a complex software component into lower-level units. The tester identifies the causes and their effects for each of the specification units. A \emph{cause} is a distinct input condition or an equivalence class of input conditions. An \emph{effect} is an output condition or a system transformation. Nodes in a Boolean cause-and-effect graph are causes and effects. Causes are placed on the left side and effects on the right side of the graph. Logical relationships between causes and effects are represented by Boolean operators AND ($\land$), OR ($\lor$), and NOT ($\thicksim$). Let's continue with an example from \autocite{burnstein2006practical}. Suppose we have a specification for a module that allows a user to perform a search for a character in an existing string. The specification states that:

\begin{displayquote}
    The user must input the length of the string and the character to search for. If the string length is out-of-range an error message will appear. If the character appears in the string, its position will be reported. If the character is not in the string the message “not found” will be output.
\end{displayquote}

The tester can identify the following causes and effects:

\begin{enumerate}[noitemsep]
    \item[C1:] Positive integer from 1 to 80
    \item[C2:] Character to search for is in string
\end{enumerate}

The effects are:

\begin{enumerate}[noitemsep]
    \item[E1:] Integer out of range
    \item[E2:] Position of character in string
    \item[E3:] Character not found
\end{enumerate}

Then, the following rules can be derived:

\begin{center}
    \begin{minipage}{0.5\textwidth}
        If C1 and C2, then E2.\\
        If C1 and not C2, then E3.\\
        If not C1, then E1.
    \end{minipage}
\end{center}

This set of rules are then converted into a cause-and-effect graph. The Figure \ref{fig:cause-n-effect-graph} shows the corresponding graph.

\begin{figure}[H]
    \centering
    \begin{tikzpicture}
        \GraphInit[vstyle=Normal]
        \Vertex[x=0, y=3]{C1}
        \Vertex[x=0, y=1]{C2}
        \Vertex[x=4, y=0]{E3}
        \Vertex[x=4, y=2]{E2}
        \Vertex[x=4, y=4]{E1}
        \Edge[label=$\thicksim$](C1)(E1)
        \Edge[label=$\thicksim$](C2)(E3)
        \Edge(C1)(E2)
        \Edge(C1)(E3)
        \Edge(C2)(E2)
        \draw (E2) node[left=0.8cm]{$\bigwedge$};
        \draw (E3) node[left=0.8cm]{$\bigwedge$};
    \end{tikzpicture}
    \caption{Cause-and-effect graph for the previously defined rules.}
    \label{fig:cause-n-effect-graph}
\end{figure}

The cause-and-effect graph can be very hard to deal with if specifications are complex enough. Because of that, the tester should convert the graph to a decision table after developing the cause-and-effect graph. This way the test cases can be inferred from the decision table instead of the graph.

\section{Decision Tables}
The decision table shows the effects of all possible combinations of causes. Each column in the decision table represents a test case and lists each combination of causes. Each row represents a cause and effect. The entries of the decision table can be a "1" for a cause or effect that is present, a "0" to represent the absence of a cause or effect, and "---" to indicate a \emph{"don't care"} value.

\begin{table}[H]
    \centering
    \renewcommand{\arraystretch}{1.2}
    \caption{Decision table for the previously defined cause-and-effect graph.}
    \label{tab:decision-table}
    \begin{tabular*}{\textwidth}{l @{\extracolsep{\fill}} llll}
        \toprule
         & \thead{T1} & \thead{T2} & \thead{T3}\\
        \midrule
        C1 & 1 & 1 & 0\\
        C2 & 1 & 0 & ---\\
        \midrule
        E1 & 0 & 0 & 1\\
        E2 & 1 & 0 & 0\\
        E3 & 0 & 1 & 0\\
        \bottomrule
    \end{tabular*}
\end{table}

The problem with decision tables is that there might be many causes and effects to consider for a complex specification. In those cases, the tester can decompose the specification into lower-level units. Then, (s)he develops cause-and-effect graphs and decision tables for these.

\section{Error Guessing}
Error guessing is based on the tester's past experience. The tester's experience with code similar to the code-under-test greatly helps her/him to find the defects. Some examples of defects that can be found by error guessing might be division by zero or conditions around array boundaries.

\section{Exercises}

\begin{exercise}
    Bank account can be 500 to 1000 for special customers,  0 to 499 for ordinary customers, 2000 for companies (the field type is integer).
    
    \begin{enumerate}[a),noitemsep]
        \item What are the equivalence classes?
        \item Fill the Table \ref{tab:ex10-question-b} by finding appropriate test cases for the equivalence classes you found in previous question (a). \emph{Add lines if necessary.}
        \item Fill the Table \ref{tab:ex10-question-c} by finding appropriate test cases for the boundary testing method. \emph{Add lines if necessary.}
    \end{enumerate}
    
    \begin{table}[H]
        \centering
        \renewcommand{\arraystretch}{1.2}
        \caption{Test cases for equivalence classes.}
        \label{tab:ex10-question-b}
        \begin{tabularx}{\textwidth}{llXX}
            \toprule
            \thead{Test Case \#} & \thead{Value} & \thead{Equivalence Classes} & \thead{Result (Val./Inval.)}\\
            \midrule
            1 & & & \\
            2 & & & \\
            3 & & & \\
            4 & & & \\
            5 & & & \\
            6 & & & \\
            7 & & & \\
            8 & & & \\
            9 & & & \\
            10 & & & \\
            11 & & & \\
            12 & & & \\
            13 & & & \\
            14 & & & \\
            \bottomrule
        \end{tabularx}
    \end{table}
    
    \begin{table}[H]
    \centering
    \renewcommand{\arraystretch}{1.2}
    \caption{Test cases for BVA strategy.}
    \label{tab:ex10-question-c}
        \begin{tabularx}{\textwidth}{llX}
            \toprule
            \thead{Test Case \#} & \thead{Value} & \thead{Result (Valid/Invalid)}\\
            \midrule
            1 & & \\
            2 & & \\
            3 & & \\
            4 & & \\
            5 & & \\
            6 & & \\
            7 & & \\
            8 & & \\
            9 & & \\
            10 & & \\
            11 & & \\
            12 & & \\
            13 & & \\
            14 & & \\
            \bottomrule
        \end{tabularx}
    \end{table}
\end{exercise}

\begin{solution}
    Answer for the item a):
    
    \begin{itemize}[noitemsep]
        \item Valid Classes
        \begin{itemize}[noitemsep]
            \item (Special) $\rightarrow$ [500, 1000]
            \item (Ordinary) $\rightarrow$ [0, 499]
            \item (Company) $\rightarrow$ 2000
        \end{itemize}
        \item Invalid Classes
        \begin{itemize}[noitemsep]
            \item (Special) $\rightarrow$ $(-\infty, 499] \cup [1001, \infty)$
            \item (Ordinary) $\rightarrow$ $(-\infty, -1] \cup [500, \infty)$
            \item (Company) $\rightarrow$ $(-\infty, 1999] \cup [2001, \infty)$
        \end{itemize}
    \end{itemize}
    
    Answer for the item b):
    
    \begin{table}[H]
    \centering
    \renewcommand{\arraystretch}{1.2}
    \caption{Suggested test cases for equivalence classes.}
    \label{tab:ex10-solution-b}
        \begin{adjustbox}{max width=\textwidth}
            \begin{tabular}{llll}
                \toprule
                \thead{Test Case \#} & \thead{Value} & \thead{Equivalence Classes} & \thead{Result (Valid/Invalid)}\\
                \midrule
                1 & 600 & (Special) $\rightarrow$ [500, 1000] & Valid\\
                2 & 300 & (Ordinary) $\rightarrow$ [0, 499] & Valid\\
                3 & 2000 & (Company) $\rightarrow$ 2000 & Valid\\
                4 & 2500 & (Special) $\rightarrow$ $(-\infty, 499] \cup [1001, \infty)$ & Invalid\\
                5 & -10 & (Ordinary) $\rightarrow$ $(-\infty, -1] \cup [500, \infty)$ & Invalid\\
                6 & 1000 & (Company) $\rightarrow$ $(-\infty, 1999] \cup [2001, \infty)$ & Invalid\\
                \bottomrule
            \end{tabular}
        \end{adjustbox}
    \end{table}
    
    \begin{table}[H]
    \centering
    \renewcommand{\arraystretch}{1.2}
    \caption{Suggested test cases for BVA strategy.}
    \label{tab:ex10-solution-c}
        \begin{tabular*}{\textwidth}{l @{\extracolsep{\fill}} lll}
            \toprule
            \thead{Test Case \#} & \thead{Value} & \thead{Result (Valid/Invalid)}\\
            \midrule
            1 & 499 & Invalid\\
            2 & 500 & Valid\\
            3 & 501 & Valid\\
            4 & 999 & Valid\\
            5 & 1000 & Valid\\
            6 & 1001 & Invalid\\
            7 & -1 & Invalid\\
            8 & 0 & Valid\\
            9 & 1 & Valid\\
            10 & 499 & Valid\\
            11 & 500 & Invalid\\
            12 & 501 & Invalid\\
            \bottomrule
        \end{tabular*}
    \end{table}
\end{solution}

\begin{exercise}
    The following is the interface of a function called \lstinline!ConvertIntToString! in the Java language.
    
    The requirements (pre-condition and post-condition) of the function are as follows:
    \begin{itemize}[noitemsep]
        \item \textbf{Pre-condition:} input is a valid int
        \item \textbf{Post-condition:} return a string corresponding to the input integer value, e.g., return string value of "-9231" for integer value of -9231. Return NULL if input is an invalid integer
    \end{itemize}

    Choose an appropriate black-box technique (equivalence class partitioning, boundary value analysis) to derive test cases for this function. Note that each test case should have a concrete input value for input \lstinline!int! and the expected output \lstinline!String!. You should use the following format for the list of your test cases.
    
    \begin{enumerate}[a),noitemsep]
        \item What are the equivalence classes? Fill the Table \ref{tab:ex11-question-a} by finding appropriate test cases for the equivalence classes.
        \begin{itemize}[noitemsep]
            \item Valid Classes:
            \item Invalid Classes:
        \end{itemize}
        \item Fill the Table \ref{tab:ex11-question-b} by finding appropriate test cases for the boundary testing method. \emph{Add lines if necessary.}
    \end{enumerate}

    \begin{table}[H]
    \centering
    \renewcommand{\arraystretch}{1.2}
    \caption{Test cases for equivalence classes.}
    \label{tab:ex11-question-a}
        \begin{tabular*}{\textwidth}{l @{\extracolsep{\fill}} llll}
            \toprule
            \thead{Test Case \#} & \thead{Value} & \thead{Equivalence Classes} & \thead{Result (Valid/Invalid)}\\
            \midrule
            1 & & & \\
            2 & & & \\
            3 & & & \\
            4 & & & \\
            \bottomrule
        \end{tabular*}
    \end{table}
    
    \begin{table}[H]
    \centering
    \renewcommand{\arraystretch}{1.2}
    \caption{Test cases for BVA strategy.}
    \label{tab:ex11-question-b}
        \begin{tabular*}{\textwidth}{l @{\extracolsep{\fill}} lll}
            \toprule
            \thead{Test Case \#} & \thead{Value} & \thead{Result (Valid/Invalid)}\\
            \midrule
            1 & & \\
            2 & & \\
            3 & & \\
            4 & & \\
            \bottomrule
        \end{tabular*}
    \end{table}
    
    \begin{lstlisting}[caption={The implementation of the program that should not supposed to be known.}]
public class IntToString {
    public static String ConvertIntToString(int number) {
        int StringConvert = 48;
        int eachDigit = number;
        int afterDivide = number;
        String reVal = "";
        
        while (afterDivide > 0) {
            eachDigit = afterDivide % 10;
            afterDivide = afterDivide / 10;
            if(eachDigit == 0) {
                reVal += "0";
            }
            else if(eachDigit == 1) {
                reVal += "1";
            }
            else if(eachDigit == 2) {
                reVal += "2";
            }
            else if(eachDigit == 3) {
                reVal += "3";
            }
            else if(eachDigit == 4) {
                reVal += "4";
            }
            else if(eachDigit == 5) {
                reVal += "5";
            }
            else if(eachDigit == 6) {
                reVal += "6";
            }
            else if(eachDigit == 7) {
                reVal += "7";
            }
            else if(eachDigit == 8) {
                reVal += "8";
            }
            else if(eachDigit == 9) {
                reVal += "9";
            }
        }
        String reVal2 = "";
        for (int index = reVal.length() -1 ; index >= 0 ; index--) {
            reVal2 += reVal.charAt(index);
        }
        return reVal2;
    }
}
    \end{lstlisting}
\end{exercise}

\begin{solution}
    Answer for the item a):
    
    \begin{itemize}[noitemsep]
        \item \textbf{Valid Classes:} (-inf, +inf) instance of integer
        \item \textbf{Invalid Classes:} String, (-inf, +inf) floating point numbers, boolean, Objects
    \end{itemize}
    
    \begin{table}[H]
    \centering
    \renewcommand{\arraystretch}{1.2}
    \caption{Suggested test cases for equivalence classes.}
    \label{tab:ex11-solution-a}
         \begin{tabular*}{\textwidth}{l @{\extracolsep{\fill}} llll}
            \toprule
            \thead{Test Case \#} & \thead{Value} & \thead{Equivalence Classes} & \thead{Result (Valid/Invalid)}\\
            \midrule
            1 & 4785 & (-inf, +inf) instance of integer & Valid\\
            2 & \lstinline!"hello"! & String & Invalid\\
            3 & 45.5 & Floating point & Invalid\\
            4 & \lstinline!'c'! & char & Invalid\\
            \bottomrule
        \end{tabular*}
    \end{table}
    
    Answer for the item b):
    \begin{table}[H]
    \centering
    \renewcommand{\arraystretch}{1.2}
    \caption{Suggested test cases for BVA strategy.}
    \label{tab:ex11-solution-b}
        \begin{tabular*}{\textwidth}{l @{\extracolsep{\fill}} lll}
            \toprule
            \thead{Test Case \#} & \thead{Value} & \thead{Result (Valid/Invalid)}\\
            \midrule
            1 & -2147483649 & Invalid\\
            2 & -2147483648 & Valid\\
            3 & 2147483647 & Valid\\
            4 & 2147483648 & Invalid\\
            \bottomrule
        \end{tabular*}
    \end{table}
\end{solution}

\begin{exercise}
    The program accepts three integers, a, b, and c as inputs. These are taken to be sides of the triangle. The integers a, b, and c must satisfy following conditions:
    
    \begin{enumerate}[label=\textbf{Condition \arabic*:},left=0pt,noitemsep]
        \item $1 \le a \le 200$
        \item $1 \le b \le 200$
        \item $1 \le c \le 200$
        \item $a < b + c$
        \item $b < a + c$
        \item $c < a + b$
    \end{enumerate}
    
    The output of the program is the type of triangle determined by the three sides: Equilateral, Isosceles, Scalene, or NotATriangle. If an input value fails any of conditions Condition 1, Condition 2 or Condition 3, the program notes this with an output message such as "Value of b is not in the range of permitted values." If values of a, b, and c satisfy Condition 4, Condition 5, and Condition 6, one of four mutually exclusive outputs is given:
    
    \begin{enumerate}[nosep]
        \item If all three sides are equal, the program output is Equilateral.
        \item If exactly one pair of sides is equal, the program output is Isosceles.
        \item If no pair of sides is equal, the program output is Scalene.
        \item If any of conditions Condition 4, Condition 5, and Condition 6 is not met, the program output is NotATriangle.
    \end{enumerate}
    
    Test the program with Decision Table-Based testing method by doing followings \autocite{jorgensen2013software}:
    \begin{enumerate}[label=\alph*),nosep]
        \item Draw the Cause-and-effect graph.
        \item Create a decision table for the problem.
        \item Create test cases.
        \item Run all test cases and write which ones are passed and which ones are failed.
    \end{enumerate}
\end{exercise}

\begin{solution}
    Answer for the item a):
    
    To draw a cause-and-effect graph, all the causes and effects should be extracted by elaborating the specification.
    
    \begin{enumerate}[nosep]
        \item[\textbf{C1:}] The given side lengths permit to build a triangle.
        \item[\textbf{C2:}] The length of side a is equal to side b.
        \item[\textbf{C3:}] The length of side b is equal to side c.
        \item[\textbf{C4:}] The length of side a is equal to side c.
        \item[\textbf{E1:}] The lengths allow to build a equilateral triangle.
        \item[\textbf{E2:}] The lengths allow to build a isosceles triangle.
        \item[\textbf{E3:}] The lengths allow to build a scalene triangle.
        \item[\textbf{E4:}] It is impossible.
        \item[\textbf{E5:}] With the given lengths, it is impossible to form a triangle.
    \end{enumerate}
    
    \begin{figure}[H]
        \centering
        \begin{tabular}{iii}
            exercise-12a-solution1 & exercise-12a-solution2 & exercise-12a-solution3\\
            exercise-12a-solution4 & exercise-12a-solution5 & exercise-12a-solution6
        \end{tabular}
        \caption{Cause-and-effect graph for the question.}
        \label{fig:cne-graphs}
    \end{figure}
    
    Answer for the item b):
    
    \begin{table}[H]
        \centering
        \renewcommand{\arraystretch}{1.2}
        \caption{Decision table for the previously defined cause-and-effect graph.}
        \label{tab:sol12-decision-table}
        \begin{tabularx}{\textwidth}{l|XXXXXXXXX}
            \toprule
             & \thead{T1} & \thead{T2} & \thead{T3} & \thead{T4} & \thead{T5} & \thead{T6} & \thead{T7} & \thead{T8} & \thead{T9}\\
            \midrule
            C1: Triangle & 0 & 1 & 1 & 1 & 1 & 1 & 1 & 1 & 1\\
            C2: a=b? & --- & 1 & 1 & 1 & 1 & 0 & 0 & 0 & 0\\
            C3: b=c? & --- & 1 & 1 & 0 & 0 & 1 & 1 & 0 & 0\\
            C4: a=c? & --- & 1 & 0 & 1 & 0 & 1 & 0 & 1 & 0\\
            \midrule
            E1: Equilateral & 0 & 1 & 0 & 0 & 0 & 0 & 0 & 0 & 0\\
            E2: Isosceles & 0 & 0 & 0 & 0 & 1 & 0 & \colorbox{red}{1} & \colorbox{red}{1} & 0\\
            E3: Scalene & 0 & 0 & 0 & 0 & 0 & 0 & 0 & 0 & \colorbox{red}{1}\\
            E4: Impossible & 0 & 0 & 1 & 1 & 0 & 1 & 0 & 0 & 0\\
            E5: Not a Triangle & 1 & 0 & 0 & 0 & 0 & 0 & 0 & 0 & 0\\
            \bottomrule
        \end{tabularx}
    \end{table}
    
    Answer for the item c):
    
    \begin{enumerate}[label=\textbf{Test \arabic*:},left=0pt,nosep]
        \item $a=1, b=2, c=7 \rightarrow E5$
        \item $a=3, b=3, c=3 \rightarrow E1$
        \item $a=2, b=2, c=3 \rightarrow E2$
        \item $a=3, b=2, c=2 \rightarrow E2$
        \item $a=2, b=3, c=2 \rightarrow E2$
        \item $a=3, b=4, c=5 \rightarrow E3$
    \end{enumerate}
    
    Answer for the item d):
    
    The red marked test cases in Table \ref{tab:sol12-decision-table} are failed tests.
\end{solution}
    % % \input{labs/07-...}
    \input{labs/08-test-automation-with-selenium-1}
    \chapter{Web Testing with Selenium 2}

\section{Examples}
The aim here is to write a program that tests another program that the tester has no idea of its code. Selenium's API can be thought of as instructions for a browser that performs the tests. First, the tester must make the test plan. Then, (s)he prepares the test cases. Finally, the test cases are converted into a Java program that is written with Selenium API to instruct the browser to perform the tests.

\subsection{Additional Exercise with Selenium}
In this exercise, you are expected to test the number of CMPE courses available on Moodle using Selenium. You should open the Moodle web page, click on “Moodle Courses”, search for “cmpe” and get the number next to the search results text. The location of the number of courses after a search is displayed in \reffig{num-search-results}. To reach that number:

\begin{figure}
    \includegraphics{images/selenium1-detailed-install-3.jpg}
    \caption{Number of search results on the Moodle web page.}
    \labfig{num-search-results}
\end{figure}

\begin{enumerate}
    \item You should get the xpath of the web element that the number is in.
    \item Using that xpath, you should get the text of that web element.
    \item Since the text will be something like “Search results: xxx” and we only need the number there, you should extract the digit part of that text.
    \item The data type of extracted digits will be string by default. So finally, to be able to store the number in an integer variable and compare with the expected result, you should convert it to integer.
\end{enumerate}

After comleting all the steps listed above, create a test class in the src/test/java folder.  Write the code below inside the class and run as Maven test.

\todo{Kod gelecek.}

\subsection{Testing Atılım University Moodle Page}
Like other universities, Atılım University has a Learning Management System (LMS) called Moodle. Moodle is an open-source LMS system that is supported by more than 600 contributors worldwide\sidenote{\url{https://github.com/moodle/moodle}}. That is all by itself guarantees that Moodle is well tested and documented. For demonstration purposes, we are going to test some of the functionality of Moodle with black-box testing strategies.

Let's start with creating a Maven project as described in Chapter \ref{lab:introduction}. This time remember to include Selenium as a dependency as described in the previous section. We will test two functionality as a warm-up practice. The first one is about testing the login link to see if the login page is successfully loaded. The second one is about checking if the SE344 course page can be searched and found through the course searching mechanism. Write the following code snippet and we will discuss the important statements later.

\lstinputlisting[language=java,caption={Testing Moodle with a few warm-up tests.},label=lst:test-moodle]{code-snippets/TestMoodle.java}

In the Listing \ref{lst:test-moodle}, we have two tests, one initialization method, and one tear-down method. The initialization method \lstinline!static void initAll()! opens and prepares a web driver which is in type \lstinline!ChromeDriver()!. This means that the driver will open a Google Chrome instance. It does that only once and assumes that WebDriver executable can be found in \lstinline[language={}]!PATH!. After that, the tests are going to run.

The \lstinline!testLogin()! test method starts with opening up the Moodle page (29). Then, it finds an anchor (link) which has a \emph{Log in} text in it via the anchor's XPath (30). XPath is a query language for querying markup languages such as HTML, XML, etc. More information about XPath can be found in Appendix \ref{ch:appendix-xpath}. After finding the link, click on it (31) and wait for a button with an id of \emph{loginbtn} to appear (32). When it appears, check if its text is \emph{Log in} (34).

The \lstinline!testSE344Course()! test method reopens the main page of Moodle (40). In line (41), we create a \lstinline!WebDriverWait! object \lstinline!wait! which we will use for explicit waits to a maximum of 10 seconds. After that, we find the \emph{Moodle Courses} link by its XPath (43) and click it (44). In the upcoming page, we choose the input field again by its XPath (46) that allows us to search \emph{SE 344} (47) and press \keys{\return} (48). After getting the desired page, we look for the text \emph{System Software Validation and Testing} on the page (50). If we find it, then the test passes.

Finally, the \lstinline!static void tearDownAll()! method run after all the tests are finished. The method checks if the driver is initialized (56) and if so, it closes it with its \lstinline!close()! method (57). This method also closes the opened browser window.

    
    \appendix
    \pagelayout{wide} % No margins
    \addpart{Appendix}
    \pagelayout{margin} % Restore margins
    
    \input{labs/10-appendix1}
    \chapter{Oracle VM VirtualBox Image}
\labch{appendix-ovb-image}
Oracle Virtual Box is used to create JUnit environment as a virtual machine to run under Windows 10 environment.  In order to create a new virtual machine, you need to install Oracle VDI and Virtual Box (\url{https://bit.ly/3w5oAEQ}) according to your choice of the OS.

A screenshot of the JUnit VB environment in Oracle VB is given in \reffig{oracleVB} below:
\begin{figure}[!ht]
    \includegraphics{images/oracleVB.png}
    \caption{Oracle Virtual Box Manager screenshot.}
    \labfig{oracleVB}
\end{figure}
    
    \backmatter % Denotes the end of the main document content
    \setchapterstyle{plain} % Output plain chapters from this point onwards
    %----------------------------------------------------------------------------------------
    %	BIBLIOGRAPHY
    %----------------------------------------------------------------------------------------
    
    % The bibliography needs to be compiled with biber using your LaTeX editor, or on the command line with 'biber main' from the template directory
    \defbibnote{bibnote}{Here are the references in citation order.\par\bigskip} % Prepend this text to the bibliography
    \printbibliography[heading=bibintoc, prenote=bibnote]
    
    %----------------------------------------------------------------------------------------
    %	NOMENCLATURE
    %----------------------------------------------------------------------------------------
    
    % The nomenclature needs to be compiled on the command line with 'makeindex main.nlo -s nomencl.ist -o main.nls' from the template directory
    
    \nomenclature{$c$}{Speed of light in a vacuum inertial frame}
    \nomenclature{$h$}{Planck constant}
    
    \renewcommand{\nomname}{Notation} % Rename the default 'Nomenclature'
    \renewcommand{\nompreamble}{The next list describes several symbols that will be later used within the body of the document.} % Prepend this text to the nomenclature
    
    \printnomenclature % Output the nomenclature
    
    %----------------------------------------------------------------------------------------
    %	GREEK ALPHABET
    % 	Originally from https://gitlab.com/jim.hefferon/linear-algebra
    %----------------------------------------------------------------------------------------
    
    \vspace{1cm}
    
    {\usekomafont{chapter}Greek Letters with Pronunciations} \\[2ex]
    \begin{center}
    	\newcommand{\pronounced}[1]{\hspace*{.2em}\small\textit{#1}}
    	\begin{tabular}{l l @{\hspace*{3em}} l l}
    		\toprule
    		Character & Name & Character & Name \\ 
    		\midrule
    		$\alpha$ & alpha \pronounced{AL-fuh} & $\nu$ & nu \pronounced{NEW} \\
    		$\beta$ & beta \pronounced{BAY-tuh} & $\xi$, $\Xi$ & xi \pronounced{KSIGH} \\ 
    		$\gamma$, $\Gamma$ & gamma \pronounced{GAM-muh} & o & omicron \pronounced{OM-uh-CRON} \\
    		$\delta$, $\Delta$ & delta \pronounced{DEL-tuh} & $\pi$, $\Pi$ & pi \pronounced{PIE} \\
    		$\epsilon$ & epsilon \pronounced{EP-suh-lon} & $\rho$ & rho \pronounced{ROW} \\
    		$\zeta$ & zeta \pronounced{ZAY-tuh} & $\sigma$, $\Sigma$ & sigma \pronounced{SIG-muh} \\
    		$\eta$ & eta \pronounced{AY-tuh} & $\tau$ & tau \pronounced{TOW (as in cow)} \\
    		$\theta$, $\Theta$ & theta \pronounced{THAY-tuh} & $\upsilon$, $\Upsilon$ & upsilon \pronounced{OOP-suh-LON} \\
    		$\iota$ & iota \pronounced{eye-OH-tuh} & $\phi$, $\Phi$ & phi \pronounced{FEE, or FI (as in hi)} \\
    		$\kappa$ & kappa \pronounced{KAP-uh} & $\chi$ & chi \pronounced{KI (as in hi)} \\
    		$\lambda$, $\Lambda$ & lambda \pronounced{LAM-duh} & $\psi$, $\Psi$ & psi \pronounced{SIGH, or PSIGH} \\
    		$\mu$ & mu \pronounced{MEW} & $\omega$, $\Omega$ & omega \pronounced{oh-MAY-guh} \\
    		\bottomrule
    	\end{tabular} \\[1.5ex]
    	Capitals shown are the ones that differ from Roman capitals.
    \end{center}
    
    %----------------------------------------------------------------------------------------
    %	GLOSSARY
    %----------------------------------------------------------------------------------------
    
    % The glossary needs to be compiled on the command line with 'makeglossaries main' from the template directory
    
    \setglossarystyle{listgroup} % Set the style of the glossary (see https://en.wikibooks.org/wiki/LaTeX/Glossary for a reference)
    \printglossary[title=Special Terms, toctitle=List of Terms] % Output the glossary, 'title' is the chapter heading for the glossary, toctitle is the table of contents heading
    
    %----------------------------------------------------------------------------------------
    %	INDEX
    %----------------------------------------------------------------------------------------
    
    % The index needs to be compiled on the command line with 'makeindex main' from the template directory
    
    \printindex % Output the index
    
    %----------------------------------------------------------------------------------------
    %	BACK COVER
    %----------------------------------------------------------------------------------------
    
    % If you have a PDF/image file that you want to use as a back cover, uncomment the following lines
    
    %\clearpage
    %\thispagestyle{empty}
    %\null%
    %\clearpage
    %\includepdf{cover-back.pdf}
    
    %----------------------------------------------------------------------------------------
\end{document}
